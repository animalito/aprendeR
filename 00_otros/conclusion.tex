\documentclass[]{article}
\usepackage{lmodern}
\usepackage{amssymb,amsmath}
\usepackage{ifxetex,ifluatex}
\usepackage{fixltx2e} % provides \textsubscript
\ifnum 0\ifxetex 1\fi\ifluatex 1\fi=0 % if pdftex
  \usepackage[T1]{fontenc}
  \usepackage[utf8]{inputenc}
\else % if luatex or xelatex
  \ifxetex
    \usepackage{mathspec}
  \else
    \usepackage{fontspec}
  \fi
  \defaultfontfeatures{Ligatures=TeX,Scale=MatchLowercase}
\fi
% use upquote if available, for straight quotes in verbatim environments
\IfFileExists{upquote.sty}{\usepackage{upquote}}{}
% use microtype if available
\IfFileExists{microtype.sty}{%
\usepackage{microtype}
\UseMicrotypeSet[protrusion]{basicmath} % disable protrusion for tt fonts
}{}
\usepackage[margin=1in]{geometry}
\usepackage{hyperref}
\hypersetup{unicode=true,
            pdftitle={Conclusión},
            pdfborder={0 0 0},
            breaklinks=true}
\urlstyle{same}  % don't use monospace font for urls
\usepackage{graphicx,grffile}
\makeatletter
\def\maxwidth{\ifdim\Gin@nat@width>\linewidth\linewidth\else\Gin@nat@width\fi}
\def\maxheight{\ifdim\Gin@nat@height>\textheight\textheight\else\Gin@nat@height\fi}
\makeatother
% Scale images if necessary, so that they will not overflow the page
% margins by default, and it is still possible to overwrite the defaults
% using explicit options in \includegraphics[width, height, ...]{}
\setkeys{Gin}{width=\maxwidth,height=\maxheight,keepaspectratio}
\IfFileExists{parskip.sty}{%
\usepackage{parskip}
}{% else
\setlength{\parindent}{0pt}
\setlength{\parskip}{6pt plus 2pt minus 1pt}
}
\setlength{\emergencystretch}{3em}  % prevent overfull lines
\providecommand{\tightlist}{%
  \setlength{\itemsep}{0pt}\setlength{\parskip}{0pt}}
\setcounter{secnumdepth}{5}
% Redefines (sub)paragraphs to behave more like sections
\ifx\paragraph\undefined\else
\let\oldparagraph\paragraph
\renewcommand{\paragraph}[1]{\oldparagraph{#1}\mbox{}}
\fi
\ifx\subparagraph\undefined\else
\let\oldsubparagraph\subparagraph
\renewcommand{\subparagraph}[1]{\oldsubparagraph{#1}\mbox{}}
\fi

%%% Use protect on footnotes to avoid problems with footnotes in titles
\let\rmarkdownfootnote\footnote%
\def\footnote{\protect\rmarkdownfootnote}

%%% Change title format to be more compact
\usepackage{titling}

% Create subtitle command for use in maketitle
\newcommand{\subtitle}[1]{
  \posttitle{
    \begin{center}\large#1\end{center}
    }
}

\setlength{\droptitle}{-2em}
  \title{Conclusión}
  \pretitle{\vspace{\droptitle}\centering\huge}
  \posttitle{\par}
  \author{}
  \preauthor{}\postauthor{}
  \date{}
  \predate{}\postdate{}

\usepackage[
  backend=biber,
  style=alphabetic,
  sorting=ynt,
  citestyle=authoryear
  ]{biblatex}
\addbibresource{../lit/bib.bib}

\usepackage[utf8]{inputenc}
\usepackage[spanish]{babel}

%%%% Frames
\ifxetex
    \makeatletter % undo the wrong changes made by mathspec
    \let\RequirePackage\original@RequirePackage
    \let\usepackage\RequirePackage
    \makeatother
\fi

\usepackage{xcolor}
\usepackage[tikz]{bclogo}
\usepackage[framemethod=tikz]{mdframed}
\usepackage{lipsum}
\usepackage[many]{tcolorbox}

\definecolor{bgblue}{RGB}{245,243,253}
\definecolor{ttblue}{RGB}{91,194,224}
\definecolor{llred}{RGB}{255,228,225}
\definecolor{bbblack}{RGB}{0,0,0}

\mdfdefinestyle{mystyle}{%
  rightline=true,
  innerleftmargin=10,
  innerrightmargin=10,
  outerlinewidth=3pt,
  topline=false,
  rightline=true,
  bottomline=false,
  skipabove=\topsep,
  skipbelow=\topsep
}

\newtcolorbox{curiosidad}[1][]{
  breakable,
  title=#1,
  colback=white,
  colbacktitle=white,
  coltitle=black,
  fonttitle=\bfseries,
  bottomrule=0pt,
  toprule=0pt,
  leftrule=3pt,
  rightrule=3pt,
  titlerule=0pt,
  arc=0pt,
  outer arc=0pt,
  colframe=black,
}

\newtcolorbox{nota}[1][]{
  breakable,
  freelance,
  title=#1,
  colback=white,
  colbacktitle=white,
  coltitle=black,
  fonttitle=\bfseries,
  bottomrule=0pt,
  boxrule=0pt,
  colframe=white,
  overlay unbroken and first={
  \draw[red!75!black,line width=3pt]
    ([xshift=5pt]frame.north west) -- 
    (frame.north west) -- 
    (frame.south west);
  \draw[red!75!black,line width=3pt]
    ([xshift=-5pt]frame.north east) -- 
    (frame.north east) -- 
    (frame.south east);
  },
  overlay unbroken app={
  \draw[red!75!black,line width=3pt,line cap=rect]
    (frame.south west) -- 
    ([xshift=5pt]frame.south west);
  \draw[red!75!black,line width=3pt,line cap=rect]
    (frame.south east) -- 
    ([xshift=-5pt]frame.south east);
  },
  overlay middle and last={
  \draw[red!75!black,line width=3pt]
    (frame.north west) -- 
    (frame.south west);
  \draw[red!75!black,line width=3pt]
    (frame.north east) -- 
    (frame.south east);
  },
  overlay last app={
  \draw[red!75!black,line width=3pt,line cap=rect]
    (frame.south west) --
    ([xshift=5pt]frame.south west);
  \draw[red!75!black,line width=3pt,line cap=rect]
    (frame.south east) --
    ([xshift=-5pt]frame.south east);
  },
}

\begin{document}


Este manual surge de la necesidad detectada de contar con material en
español que permita al usuario aprender a operar un proyecto de datos,
más allá del aprendizaje del lenguaje de programación. A través de cada
capitulo, se espera que el lector se familiarice con los elementos
básicos del ambiente de \texttt{R}, sus más importantes estructuras de
datos, así como distintas maneras de operarlas. Sin embargo, el enfoque
del manual es que el lector cuente con las herramientas necesarias para
poder desarrollar un proyecto de datos.

El manual está estructurado para que en los primeros tres capítulos el
lector conozca las principales herramientas que facilitan el trabajo del
analista a través del dominio básico de la sintaxis de \texttt{R}, para,
posteriormente, en el capítulo 4, centrarse en el ciclo de análisis de
datos.

Para éste ciclo, se revisaron los métodos de importación de datos al
ambiente de \texttt{R}, la manipulación y limpieza de datos, revisando a
fondo los verbos implementados en \texttt{dplyr} y el concepto de datos
limpios, así como su implementación en \texttt{tidyr}. Por último, para
concluir el ciclo de análisis se mostró una introducción a la gramática
de gráficas y su implementación en \texttt{ggplot2}. Se utilizó la
conjugación de los verbos de \texttt{dplyr}, pues estos permiten
realizar casi todas las operaciones que un analista debe realizar para
preparar sus datos para el modelado, facilitando su visualización o
presentación en formato tabular, así como permitir la realización de
estadística descriptiva.

Por último, con la finalidad de que el lector pueda practicar lo
aprendido se incluyeron múltiples ejercicios y referencias a otro tipo
de herramientas que salen del alcance de este manual pero que
complementan al material presentado, como lo son, las expresiones
regulares, el trabajo con fechas, la manipulación de cadenas de
caracteres, entre otros.

A través de diversas iteraciones con distintos grupos, el material ha
probado ser útil como guía en varios cursos introductorios a \texttt{R},
cumpliendo con el objetivo o alcance del mismo.

Con la finalidad de fortalecer el presente manual, así como seguir
construyendo material en español que permita el desarrollo de proyectos
de datos cada vez más sólidos, es posible ampliar el material de manera
que se cubran distintas herramientas para el modelado, particularmente
las implementadas en \texttt{purrr} que continúan con el
\emph{framework} de datos limpios para esta etapa en el ciclo de
análisis; así como ahondar en el apartado de comunicación, que solo se
cubrió en el capítulo 2 presentando \texttt{rmarkdown} para compilar
documentos reproducibles, así como traducir los cursos de \texttt{swirl}
a los que se hace referencia.

En este sentido, se invita a que el lector proporcione comentarios y
sugerencias para mejorar o ampliar el material a través de \emph{issues}
o \emph{pull requests} en el repositorio del proyecto:
\url{https://github.com/animalito/aprendeR}.


\end{document}
