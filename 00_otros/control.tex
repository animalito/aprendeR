\documentclass[]{article}
\usepackage{lmodern}
\usepackage{amssymb,amsmath}
\usepackage{ifxetex,ifluatex}
\usepackage{fixltx2e} % provides \textsubscript
\ifnum 0\ifxetex 1\fi\ifluatex 1\fi=0 % if pdftex
  \usepackage[T1]{fontenc}
  \usepackage[utf8]{inputenc}
\else % if luatex or xelatex
  \ifxetex
    \usepackage{mathspec}
  \else
    \usepackage{fontspec}
  \fi
  \defaultfontfeatures{Ligatures=TeX,Scale=MatchLowercase}
\fi
% use upquote if available, for straight quotes in verbatim environments
\IfFileExists{upquote.sty}{\usepackage{upquote}}{}
% use microtype if available
\IfFileExists{microtype.sty}{%
\usepackage{microtype}
\UseMicrotypeSet[protrusion]{basicmath} % disable protrusion for tt fonts
}{}
\usepackage[margin=1in]{geometry}
\usepackage{hyperref}
\hypersetup{unicode=true,
            pdftitle={Packrat},
            pdfborder={0 0 0},
            breaklinks=true}
\urlstyle{same}  % don't use monospace font for urls
\usepackage{color}
\usepackage{fancyvrb}
\newcommand{\VerbBar}{|}
\newcommand{\VERB}{\Verb[commandchars=\\\{\}]}
\DefineVerbatimEnvironment{Highlighting}{Verbatim}{commandchars=\\\{\}}
% Add ',fontsize=\small' for more characters per line
\usepackage{framed}
\definecolor{shadecolor}{RGB}{248,248,248}
\newenvironment{Shaded}{\begin{snugshade}}{\end{snugshade}}
\newcommand{\KeywordTok}[1]{\textcolor[rgb]{0.13,0.29,0.53}{\textbf{#1}}}
\newcommand{\DataTypeTok}[1]{\textcolor[rgb]{0.13,0.29,0.53}{#1}}
\newcommand{\DecValTok}[1]{\textcolor[rgb]{0.00,0.00,0.81}{#1}}
\newcommand{\BaseNTok}[1]{\textcolor[rgb]{0.00,0.00,0.81}{#1}}
\newcommand{\FloatTok}[1]{\textcolor[rgb]{0.00,0.00,0.81}{#1}}
\newcommand{\ConstantTok}[1]{\textcolor[rgb]{0.00,0.00,0.00}{#1}}
\newcommand{\CharTok}[1]{\textcolor[rgb]{0.31,0.60,0.02}{#1}}
\newcommand{\SpecialCharTok}[1]{\textcolor[rgb]{0.00,0.00,0.00}{#1}}
\newcommand{\StringTok}[1]{\textcolor[rgb]{0.31,0.60,0.02}{#1}}
\newcommand{\VerbatimStringTok}[1]{\textcolor[rgb]{0.31,0.60,0.02}{#1}}
\newcommand{\SpecialStringTok}[1]{\textcolor[rgb]{0.31,0.60,0.02}{#1}}
\newcommand{\ImportTok}[1]{#1}
\newcommand{\CommentTok}[1]{\textcolor[rgb]{0.56,0.35,0.01}{\textit{#1}}}
\newcommand{\DocumentationTok}[1]{\textcolor[rgb]{0.56,0.35,0.01}{\textbf{\textit{#1}}}}
\newcommand{\AnnotationTok}[1]{\textcolor[rgb]{0.56,0.35,0.01}{\textbf{\textit{#1}}}}
\newcommand{\CommentVarTok}[1]{\textcolor[rgb]{0.56,0.35,0.01}{\textbf{\textit{#1}}}}
\newcommand{\OtherTok}[1]{\textcolor[rgb]{0.56,0.35,0.01}{#1}}
\newcommand{\FunctionTok}[1]{\textcolor[rgb]{0.00,0.00,0.00}{#1}}
\newcommand{\VariableTok}[1]{\textcolor[rgb]{0.00,0.00,0.00}{#1}}
\newcommand{\ControlFlowTok}[1]{\textcolor[rgb]{0.13,0.29,0.53}{\textbf{#1}}}
\newcommand{\OperatorTok}[1]{\textcolor[rgb]{0.81,0.36,0.00}{\textbf{#1}}}
\newcommand{\BuiltInTok}[1]{#1}
\newcommand{\ExtensionTok}[1]{#1}
\newcommand{\PreprocessorTok}[1]{\textcolor[rgb]{0.56,0.35,0.01}{\textit{#1}}}
\newcommand{\AttributeTok}[1]{\textcolor[rgb]{0.77,0.63,0.00}{#1}}
\newcommand{\RegionMarkerTok}[1]{#1}
\newcommand{\InformationTok}[1]{\textcolor[rgb]{0.56,0.35,0.01}{\textbf{\textit{#1}}}}
\newcommand{\WarningTok}[1]{\textcolor[rgb]{0.56,0.35,0.01}{\textbf{\textit{#1}}}}
\newcommand{\AlertTok}[1]{\textcolor[rgb]{0.94,0.16,0.16}{#1}}
\newcommand{\ErrorTok}[1]{\textcolor[rgb]{0.64,0.00,0.00}{\textbf{#1}}}
\newcommand{\NormalTok}[1]{#1}
\usepackage{graphicx,grffile}
\makeatletter
\def\maxwidth{\ifdim\Gin@nat@width>\linewidth\linewidth\else\Gin@nat@width\fi}
\def\maxheight{\ifdim\Gin@nat@height>\textheight\textheight\else\Gin@nat@height\fi}
\makeatother
% Scale images if necessary, so that they will not overflow the page
% margins by default, and it is still possible to overwrite the defaults
% using explicit options in \includegraphics[width, height, ...]{}
\setkeys{Gin}{width=\maxwidth,height=\maxheight,keepaspectratio}
\IfFileExists{parskip.sty}{%
\usepackage{parskip}
}{% else
\setlength{\parindent}{0pt}
\setlength{\parskip}{6pt plus 2pt minus 1pt}
}
\setlength{\emergencystretch}{3em}  % prevent overfull lines
\providecommand{\tightlist}{%
  \setlength{\itemsep}{0pt}\setlength{\parskip}{0pt}}
\setcounter{secnumdepth}{0}
% Redefines (sub)paragraphs to behave more like sections
\ifx\paragraph\undefined\else
\let\oldparagraph\paragraph
\renewcommand{\paragraph}[1]{\oldparagraph{#1}\mbox{}}
\fi
\ifx\subparagraph\undefined\else
\let\oldsubparagraph\subparagraph
\renewcommand{\subparagraph}[1]{\oldsubparagraph{#1}\mbox{}}
\fi

%%% Use protect on footnotes to avoid problems with footnotes in titles
\let\rmarkdownfootnote\footnote%
\def\footnote{\protect\rmarkdownfootnote}

%%% Change title format to be more compact
\usepackage{titling}

% Create subtitle command for use in maketitle
\newcommand{\subtitle}[1]{
  \posttitle{
    \begin{center}\large#1\end{center}
    }
}

\setlength{\droptitle}{-2em}
  \title{Packrat}
  \pretitle{\vspace{\droptitle}\centering\huge}
  \posttitle{\par}
  \author{}
  \preauthor{}\postauthor{}
  \date{}
  \predate{}\postdate{}

\usepackage[
  backend=biber,
  style=alphabetic,
  sorting=ynt,
  citestyle=authoryear
  ]{biblatex}
\addbibresource{../lit/bib.bib}

\usepackage[utf8]{inputenc}
\usepackage[spanish]{babel}

%%%% Frames
\ifxetex
    \makeatletter % undo the wrong changes made by mathspec
    \let\RequirePackage\original@RequirePackage
    \let\usepackage\RequirePackage
    \makeatother
\fi

\usepackage{xcolor}
\usepackage[tikz]{bclogo}
\usepackage[framemethod=tikz]{mdframed}
\usepackage{lipsum}
\usepackage[many]{tcolorbox}

\definecolor{bgblue}{RGB}{245,243,253}
\definecolor{ttblue}{RGB}{91,194,224}
\definecolor{llred}{RGB}{255,228,225}
\definecolor{bbblack}{RGB}{0,0,0}

\mdfdefinestyle{mystyle}{%
  rightline=true,
  innerleftmargin=10,
  innerrightmargin=10,
  outerlinewidth=3pt,
  topline=false,
  rightline=true,
  bottomline=false,
  skipabove=\topsep,
  skipbelow=\topsep
}

\newtcolorbox{curiosidad}[1][]{
  breakable,
  title=#1,
  colback=white,
  colbacktitle=white,
  coltitle=black,
  fonttitle=\bfseries,
  bottomrule=0pt,
  toprule=0pt,
  leftrule=3pt,
  rightrule=3pt,
  titlerule=0pt,
  arc=0pt,
  outer arc=0pt,
  colframe=black,
}

\newtcolorbox{nota}[1][]{
  breakable,
  freelance,
  title=#1,
  colback=white,
  colbacktitle=white,
  coltitle=black,
  fonttitle=\bfseries,
  bottomrule=0pt,
  boxrule=0pt,
  colframe=white,
  overlay unbroken and first={
  \draw[red!75!black,line width=3pt]
    ([xshift=5pt]frame.north west) -- 
    (frame.north west) -- 
    (frame.south west);
  \draw[red!75!black,line width=3pt]
    ([xshift=-5pt]frame.north east) -- 
    (frame.north east) -- 
    (frame.south east);
  },
  overlay unbroken app={
  \draw[red!75!black,line width=3pt,line cap=rect]
    (frame.south west) -- 
    ([xshift=5pt]frame.south west);
  \draw[red!75!black,line width=3pt,line cap=rect]
    (frame.south east) -- 
    ([xshift=-5pt]frame.south east);
  },
  overlay middle and last={
  \draw[red!75!black,line width=3pt]
    (frame.north west) -- 
    (frame.south west);
  \draw[red!75!black,line width=3pt]
    (frame.north east) -- 
    (frame.south east);
  },
  overlay last app={
  \draw[red!75!black,line width=3pt,line cap=rect]
    (frame.south west) --
    ([xshift=5pt]frame.south west);
  \draw[red!75!black,line width=3pt,line cap=rect]
    (frame.south east) --
    ([xshift=-5pt]frame.south east);
  },
}

\begin{document}


Uno de los problemas en el trabajo colaborativo es poder ejecutar código
realizado en otra computadora. La analítica reproducible y fácil de
insertar en un ambiente de producción es fundamental para minimizar el
retrabajo y que lo que se realice se (re)utilice.

Existen múltiples maneras de trabajar de manera que se resuelva el
problema de las versiones de software y sus dependencias. Una
comprehensiva, por ejemplo, es usando \texttt{docker}. Cuando un
proyecto incluye únicamente código de \texttt{R}, \texttt{packrat} es
suficiente para empaquetarlo y que el código sea reproducible en
cualquier computadora y sistema operativo \parencite{packrat}.

\href{https://rstudio.github.io/packrat/}{Packrat} es un sistema de
administración de dependencias para R que busca eliminar los problemas
que suele haber para utilizar código realizado en diferentes momentos o
máquinas con diferentes versiones de las librerías, entre los típicos
son \parencite{packratconcept}:

\begin{itemize}
\tightlist
\item
  La falta de control sobre los paquetes que se necesitaban instalar
  para correr un script específico.
\item
  Instalar paquetes en el ambiente global y dejarlos para siempre
  instalados en las computadoras porque no se sabe si algo se romperá al
  quitarlos.
\item
  Romper código de otros proyectos por actualizar un paquete en otro.
\end{itemize}

Packrat permite que los proyectos en \texttt{R} sean
\parencite{packratconcept}:

\begin{itemize}
\tightlist
\item
  \textbf{Aislados}: cada proyecto tiene su paquetería privada.
\item
  \textbf{Portables}: puedes transferir rápidamente los proyectos de una
  computadora a otra -y a través de distintas plataformas- pues facilita
  la instalación de toda la paquetería sobre la que descansa el
  proyecto.
\item
  \textbf{Reproducibles}: guarda las versiones exactas sobre las que el
  proyecto fue trabajado y éstos son los que son instalados en cualquier
  ambiente.
\end{itemize}

\section{El directorio del proyecto}\label{el-directorio-del-proyecto}

Packrat se asocia a un directorio específico. Al iniciar una sesión de R
dentro de un directorio asociado a un packrat, R va a utilizar
únicamente los paquetes dentro de esa librería privada. Al instalar,
remover o actualizar un paquete dentro de ese directorio, esos cambios
se harán en la librería privada.

Se guarda en el proyecto toda la información que packrat necesita para
poder recrear el conjunto de librerías en cualquier otra máquina.

\section{Instalación}\label{instalacion}

El paquete está en el CRAN y se instala desde R con el comando.

\begin{Shaded}
\begin{Highlighting}[]
\KeywordTok{install.packages}\NormalTok{(}\StringTok{"packrat"}\NormalTok{)}
\end{Highlighting}
\end{Shaded}

\section{Inicializarlo}\label{inicializarlo}

Al iniciar un proyecto, el que sea, que use R, lo recomendable es
asociarle packrat. Esto se hace con el comando \texttt{packrat::init}.

\begin{Shaded}
\begin{Highlighting}[]
\NormalTok{packrat}\OperatorTok{::}\KeywordTok{init}\NormalTok{(}\StringTok{"~/prueba-packrat"}\NormalTok{)}
\end{Highlighting}
\end{Shaded}

Con esto, al trabajar en el directorio
\texttt{"\textasciitilde{}/prueba-packrat"} ya estás en un proyecto de
packrat con su librería privada.

Un proyecto de packrat se distingue porque -igual que git- tiene
archivos y directorios adicionales que se crean con la función
\texttt{init()}:

\begin{itemize}
\tightlist
\item
  \texttt{packrat/packrat.lock}: lista las versiones de los paquetes que
  fueron utilizadas. Este archivo no debe editarse a mano.
\item
  \texttt{packrat/packrat.opts}: guarda las opciones de configuración
  para el proyecto. Este se puede modificar con las opciones
  \texttt{get\_opts} y \texttt{set\_opts}. La lista completa de opciones
  se puede ver al escribir en la consola de R
  \texttt{?"packrat-options"}.
\item
  \texttt{packrat/lib/}: paquetes para el proyecto.
\item
  \texttt{packrat/src/}: paquetes para todas las dependencias.
\item
  \texttt{.Rprofile}: Le dice a R que la lista específica de librerías
  que debe utilizar cuando está en ese directorio (o cualquiera de sus
  subdirectorios) es la privada del proyecto que gestiona packrat.
\end{itemize}

\section{Agregar, remover y actualizar
paquetes}\label{agregar-remover-y-actualizar-paquetes}

\begin{enumerate}
\def\labelenumi{\arabic{enumi}.}
\tightlist
\item
  Inicializa \texttt{R} dentro de un proyecto \texttt{packrat}.
\item
  Se instala como siempre, usando \texttt{install.packages()}
\end{enumerate}

\begin{Shaded}
\begin{Highlighting}[]
\KeywordTok{install.packages}\NormalTok{(}\StringTok{"dplyr"}\NormalTok{)}
\end{Highlighting}
\end{Shaded}

\begin{enumerate}
\def\labelenumi{\arabic{enumi}.}
\setcounter{enumi}{2}
\tightlist
\item
  Se toma un \texttt{snapshot} para decirle a packrat que guarde los
  cambios
\end{enumerate}

\begin{Shaded}
\begin{Highlighting}[]
\NormalTok{packrat}\OperatorTok{::}\KeywordTok{snapshot}\NormalTok{()}
\end{Highlighting}
\end{Shaded}

Aquí \texttt{packrat} agrega lo que necesita a los folders mencionados
antes para poder recrear las versiones y dependencias. También modifica
el archivo \texttt{packrat.lock}.

\begin{enumerate}
\def\labelenumi{\arabic{enumi}.}
\setcounter{enumi}{3}
\tightlist
\item
  En cualquier momento, puedes revisar el estatus
\end{enumerate}

\begin{Shaded}
\begin{Highlighting}[]
\NormalTok{packrat}\OperatorTok{::}\KeywordTok{status}\NormalTok{()}
\end{Highlighting}
\end{Shaded}

Debe darte el mensaje \textbf{Up to date}.

¡Y listo!

\section{Otras cosas importantes}\label{otras-cosas-importantes}

\begin{itemize}
\tightlist
\item
  Packrat puede incorporar paquetes que no están en CRAN.
\item
  Puede restaurar un snapshot (de manera similar a la que un juego
  puedes regresar al último checkpoint).
\end{itemize}

\renewcommand\bcStyleTitre[1]{\large\textcolor{bbblack}{#1}}

\begin{bclogo}[
  couleur=llred,
  arrondi=0,
  logo=\bcstop,
  barre=none,
  noborder=true]{Ejercicio: Restaurando un snapshot en el ejemplo de juguete.}
\begin{enumerate}
\item Ve a la carpeta \texttt{~/prueba-packrat}
\item Muevete a la carpeta \texttt{packrat/}
\item Borra la libreria \texttt{rm -R lib/}
\item Regresa a la carpeta del proyecto \texttt{cd ..}
\item Inicializa \texttt{R}... y todo se restaura.
\end{enumerate}
\end{bclogo}

\section{Ligas utiles}\label{ligas-utiles}

\begin{itemize}
\tightlist
\item
  \href{https://rstudio.github.io/packrat/commands.html}{Comandos
  comunes y listas de opciones}
\item
  \href{https://rstudio.github.io/packrat/rstudio.html}{Packrat en
  RStudio} lo hace AUN mas fácil.
\item
  \href{https://rstudio.github.io/packrat/limitations.html}{Limitaciones
  de packrat}
\end{itemize}


\end{document}
