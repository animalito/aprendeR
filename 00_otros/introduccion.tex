\documentclass[]{article}
\usepackage{lmodern}
\usepackage{amssymb,amsmath}
\usepackage{ifxetex,ifluatex}
\usepackage{fixltx2e} % provides \textsubscript
\ifnum 0\ifxetex 1\fi\ifluatex 1\fi=0 % if pdftex
  \usepackage[T1]{fontenc}
  \usepackage[utf8]{inputenc}
\else % if luatex or xelatex
  \ifxetex
    \usepackage{mathspec}
  \else
    \usepackage{fontspec}
  \fi
  \defaultfontfeatures{Ligatures=TeX,Scale=MatchLowercase}
\fi
% use upquote if available, for straight quotes in verbatim environments
\IfFileExists{upquote.sty}{\usepackage{upquote}}{}
% use microtype if available
\IfFileExists{microtype.sty}{%
\usepackage{microtype}
\UseMicrotypeSet[protrusion]{basicmath} % disable protrusion for tt fonts
}{}
\usepackage[margin=1in]{geometry}
\usepackage{hyperref}
\hypersetup{unicode=true,
            pdftitle={Introducción},
            pdfborder={0 0 0},
            breaklinks=true}
\urlstyle{same}  % don't use monospace font for urls
\usepackage{graphicx,grffile}
\makeatletter
\def\maxwidth{\ifdim\Gin@nat@width>\linewidth\linewidth\else\Gin@nat@width\fi}
\def\maxheight{\ifdim\Gin@nat@height>\textheight\textheight\else\Gin@nat@height\fi}
\makeatother
% Scale images if necessary, so that they will not overflow the page
% margins by default, and it is still possible to overwrite the defaults
% using explicit options in \includegraphics[width, height, ...]{}
\setkeys{Gin}{width=\maxwidth,height=\maxheight,keepaspectratio}
\IfFileExists{parskip.sty}{%
\usepackage{parskip}
}{% else
\setlength{\parindent}{0pt}
\setlength{\parskip}{6pt plus 2pt minus 1pt}
}
\setlength{\emergencystretch}{3em}  % prevent overfull lines
\providecommand{\tightlist}{%
  \setlength{\itemsep}{0pt}\setlength{\parskip}{0pt}}
\setcounter{secnumdepth}{5}
% Redefines (sub)paragraphs to behave more like sections
\ifx\paragraph\undefined\else
\let\oldparagraph\paragraph
\renewcommand{\paragraph}[1]{\oldparagraph{#1}\mbox{}}
\fi
\ifx\subparagraph\undefined\else
\let\oldsubparagraph\subparagraph
\renewcommand{\subparagraph}[1]{\oldsubparagraph{#1}\mbox{}}
\fi

%%% Use protect on footnotes to avoid problems with footnotes in titles
\let\rmarkdownfootnote\footnote%
\def\footnote{\protect\rmarkdownfootnote}

%%% Change title format to be more compact
\usepackage{titling}

% Create subtitle command for use in maketitle
\newcommand{\subtitle}[1]{
  \posttitle{
    \begin{center}\large#1\end{center}
    }
}

\setlength{\droptitle}{-2em}
  \title{Introducción}
  \pretitle{\vspace{\droptitle}\centering\huge}
  \posttitle{\par}
  \author{}
  \preauthor{}\postauthor{}
  \date{}
  \predate{}\postdate{}

\usepackage[
  backend=biber,
  style=alphabetic,
  sorting=ynt,
  citestyle=authoryear
  ]{biblatex}
\addbibresource{../lit/bib.bib}

\usepackage[utf8]{inputenc}
\usepackage[spanish]{babel}

%%%% Frames
\ifxetex
    \makeatletter % undo the wrong changes made by mathspec
    \let\RequirePackage\original@RequirePackage
    \let\usepackage\RequirePackage
    \makeatother
\fi

\usepackage{xcolor}
\usepackage[tikz]{bclogo}
\usepackage[framemethod=tikz]{mdframed}
\usepackage{lipsum}
\usepackage[many]{tcolorbox}

\definecolor{bgblue}{RGB}{245,243,253}
\definecolor{ttblue}{RGB}{91,194,224}
\definecolor{llred}{RGB}{255,228,225}
\definecolor{bbblack}{RGB}{0,0,0}

\mdfdefinestyle{mystyle}{%
  rightline=true,
  innerleftmargin=10,
  innerrightmargin=10,
  outerlinewidth=3pt,
  topline=false,
  rightline=true,
  bottomline=false,
  skipabove=\topsep,
  skipbelow=\topsep
}

\newtcolorbox{curiosidad}[1][]{
  breakable,
  title=#1,
  colback=white,
  colbacktitle=white,
  coltitle=black,
  fonttitle=\bfseries,
  bottomrule=0pt,
  toprule=0pt,
  leftrule=3pt,
  rightrule=3pt,
  titlerule=0pt,
  arc=0pt,
  outer arc=0pt,
  colframe=black,
}

\newtcolorbox{nota}[1][]{
  breakable,
  freelance,
  title=#1,
  colback=white,
  colbacktitle=white,
  coltitle=black,
  fonttitle=\bfseries,
  bottomrule=0pt,
  boxrule=0pt,
  colframe=white,
  overlay unbroken and first={
  \draw[red!75!black,line width=3pt]
    ([xshift=5pt]frame.north west) -- 
    (frame.north west) -- 
    (frame.south west);
  \draw[red!75!black,line width=3pt]
    ([xshift=-5pt]frame.north east) -- 
    (frame.north east) -- 
    (frame.south east);
  },
  overlay unbroken app={
  \draw[red!75!black,line width=3pt,line cap=rect]
    (frame.south west) -- 
    ([xshift=5pt]frame.south west);
  \draw[red!75!black,line width=3pt,line cap=rect]
    (frame.south east) -- 
    ([xshift=-5pt]frame.south east);
  },
  overlay middle and last={
  \draw[red!75!black,line width=3pt]
    (frame.north west) -- 
    (frame.south west);
  \draw[red!75!black,line width=3pt]
    (frame.north east) -- 
    (frame.south east);
  },
  overlay last app={
  \draw[red!75!black,line width=3pt,line cap=rect]
    (frame.south west) --
    ([xshift=5pt]frame.south west);
  \draw[red!75!black,line width=3pt,line cap=rect]
    (frame.south east) --
    ([xshift=-5pt]frame.south east);
  },
}

\begin{document}


\texttt{R} inicia a principios de los noventas en la Universidad de
Auckland en Nueva Zelanda. Ross Ihaka, profesor del departamento de
estadística, pensaba que debía existir una alternativa superior para el
análisis de datos realizado por los alumnos, que utilizaban lo que él
llamaba \emph{programas viejos y cuchos}. Robert Gentleman le sugiere a
Ross escribir un software cuya ambición inicial era poder enseñar sus
cursos de licenciatura de primer año. Así, en 1991 generan una
estructura básica a través de la cuál sus estudiantes podían hacer
análisis de datos y producir modelos gráficos de la información. Lo
bautizan \texttt{R} por sus iniciales \parencite{rorigins}.

Ross y Robert no comercializan el software sino que lo ponen a
disposición de otros interesados. Ross ha expresado que \texttt{R}
cambió su opinión acerca de la humanidad pues es el resultado del
trabajo de muchos que no reciben ingresos o reconocimiento por el mismo
\parencite{rorigins}. En 1996, presentan \texttt{R} en un paper
introductorio \parencite{ihaka1996r}.

A partir de entonces, \texttt{R} ha crecido en forma importante. Entre
los contribuidores actuales más relevantes se encuentra Hadley Wickham,
alumno de licenciatura en el departamento de estadística de la
Universidad de Auckland cuando \texttt{R} se encontraba en desarrollo.
En la gráfica siguiente, se muestran las descargas anuales de paquetes
de \texttt{R} del 2012 al 2016 del espejo de
RStudio\footnote{Estos números representan únicamente una fracción de las descargas de \texttt{R} en el mundo pues existen múltiples espejos del software de donde es posible realizar la descarga. Los datos son tomados de \textcite{cranlogs}}.

\begin{figure}
\centering
\includegraphics{introduccion_files/figure-latex/unnamed-chunk-1-1.pdf}
\caption{Descargas anuales del espejo de RStudio de paquetes de R de
2012 a 2016 y descargas de R para 2015 y 2016 (en millones).}
\end{figure}

En el 2016 \texttt{R} fue descargado 670,705 veces. El aumento en la
popularidad de \texttt{R} no es el único elemento por el cuál \texttt{R}
es un lenguaje valioso. Sin embargo, el que sea un lenguaje comúnmente
enseñado en universidades y utilizado en empresas, lo convierte en una
habilidad con considerable valor de mercado.

En la encuesta de \texttt{Stackoverflow}, \texttt{R} se encuentra en el
lugar séptimo de los mejores pagados para los desarrolladores cuya
ocupación es matemáticas, superando a \texttt{Python} y a \texttt{SQL}
\parencite[][Top paying tech per occupation, mathematics]{stackoverflowsurvey16}.
En cuanto a las tecnologías más populares por tipo de desarrollador que
declara dedicarse a matemáticas y datos, \texttt{R} está en el sexto
lugar, el primer lugar lo tiene \texttt{python}, seguido de \texttt{SQL}
\parencite[][Most Popular Technologies per Dev Type, Math and Data]{stackoverflowsurvey16}.

Actualmente, \texttt{R}, \texttt{python} y \texttt{SQL} se encuentran
entre las herramientas más populares tanto entre desarrolladores como
empresas, aunque no son las únicas. La decisión de aprender alguno de
estos lenguajes depende de muchos factores, entre ellos cuán natural
resulta la interacción individual con cada cuál, el lenguaje preferido
en el grupo de trabajo particular y el tipo de análisis que se requiere
realizar en el día a día. Escapa del objetivo de este manual el realizar
una comparación exhaustiva de tecnologías pero se recomienda tener en
cuenta que cada herramienta tiene una especialidad específica y,
particularmente en un ambiente de producción, es necesario tener esto en
consideración.

\texttt{R} es un excelente lenguaje para aprender ciencia de datos; de
hecho en \textcite{cran} se describe a \texttt{R} como un proyecto para
estadística computacional. Esto lo convierte en un lenguaje único pues
fue construido por estadísticos y diseñado para realizar análisis de
datos.

Su uso generalizado en la comunidad estadística tiene la ventaja de que
casi cualquier prueba o técnica estadística puede ser encontrada en
algún paquete de \texttt{R} \parencite{recommendr}. Además, existe una
documentación extensa y estandarizada que facilita su uso.

Aunque el material para aprender \texttt{R} es amplio y hay una
comunidad mundial muy activa que constantemente produce nuevos recursos,
existen pocas referencias que faciliten iniciar su aprendizaje para
hispanoparlantes. En general, la documentación, listas de distribución,
libros y tutoriales están escritos en inglés.

Este manual tiene como objetivo guiar a principiantes en programación
que tienen una formación previa como analistas de datos. El enfoque
principal es el de facilitar de ejemplos que permitan al analista
traducir la manipulación de datos que ya saben realizar en otro ambiente
a \texttt{R}.

El manual se estructura como sigue: en el capítulo 2, se introducen
elementos básicos para poder iniciar el trabajo en \texttt{R}. Se
especifica cómo instalar el software, se recomienda utilizar un editor
especializado, así como paquetes útiles para diferentes tareas. En
particular, se explica cómo guardar código de manera que otras personas
puedan ejecutarlo y cómo realizar documentos reproducibles. Por último,
se explica cómo accesar a la ayuda y documentación, así como la forma en
la que puede optimizarse su funcionamiento. Este capítulo actúa más como
una referencia general para poder realizar el trabajo en el ambiente.

En el capítulo 3, se introducen las funciones, las estructuras de datos
y las estructuras de control disponibles en el lenguaje. El capítulo 4,
explica como operar los objetos y estructuras detallados en el capítulo
anterior, proporcionando múltiples ejemplos y ejercicios para
familiarizar al lector con el lenguaje.

El capítulo 5, detalla las herramientas básicas para poder realizar un
proyecto de datos en \texttt{R}. Las herramientas que se desarrollan en
este capítulo permiten iterar sobre parte del ciclo de un proyecto de
datos: importación de datos al ambiente, manipulación, limpieza y
visualización de los mismos. Éstas herramientas permiten operar sobre
los objetos introducidos en el capítulo 3 en una forma eficiente, fácil
de aprender, fácil de leer y que permite que el usuario realice
manipulaciones de datos complejas que le permitirán, a su vez, utilizar
todas las herramientas de modelado que \texttt{R} posee que necesitan
como insumo datos limpios y preparados en una forma específica.

Cada capítulo incluye ejercicios y respuestas a los mismos; al final se
recomienda material adicional para repasar los conceptos estudiados. El
material se encuentra disponible electrónicamente en
\url{https://github.com/animalito/aprendeR}. Para facilitar el
aprendizaje, se recomienda descargar los materiales o clonar el
repositorio, esto permite revisar el material y el código desde el
ambiente local evitando copiar y pegar el mismo para su ejecución.


\end{document}
