\documentclass[]{article}
\usepackage{lmodern}
\usepackage{amssymb,amsmath}
\usepackage{ifxetex,ifluatex}
\usepackage{fixltx2e} % provides \textsubscript
\ifnum 0\ifxetex 1\fi\ifluatex 1\fi=0 % if pdftex
  \usepackage[T1]{fontenc}
  \usepackage[utf8]{inputenc}
\else % if luatex or xelatex
  \ifxetex
    \usepackage{mathspec}
  \else
    \usepackage{fontspec}
  \fi
  \defaultfontfeatures{Ligatures=TeX,Scale=MatchLowercase}
\fi
% use upquote if available, for straight quotes in verbatim environments
\IfFileExists{upquote.sty}{\usepackage{upquote}}{}
% use microtype if available
\IfFileExists{microtype.sty}{%
\usepackage{microtype}
\UseMicrotypeSet[protrusion]{basicmath} % disable protrusion for tt fonts
}{}
\usepackage[margin=1in]{geometry}
\usepackage{hyperref}
\hypersetup{unicode=true,
            pdftitle={Markdown},
            pdfborder={0 0 0},
            breaklinks=true}
\urlstyle{same}  % don't use monospace font for urls
\usepackage{color}
\usepackage{fancyvrb}
\newcommand{\VerbBar}{|}
\newcommand{\VERB}{\Verb[commandchars=\\\{\}]}
\DefineVerbatimEnvironment{Highlighting}{Verbatim}{commandchars=\\\{\}}
% Add ',fontsize=\small' for more characters per line
\usepackage{framed}
\definecolor{shadecolor}{RGB}{248,248,248}
\newenvironment{Shaded}{\begin{snugshade}}{\end{snugshade}}
\newcommand{\KeywordTok}[1]{\textcolor[rgb]{0.13,0.29,0.53}{\textbf{#1}}}
\newcommand{\DataTypeTok}[1]{\textcolor[rgb]{0.13,0.29,0.53}{#1}}
\newcommand{\DecValTok}[1]{\textcolor[rgb]{0.00,0.00,0.81}{#1}}
\newcommand{\BaseNTok}[1]{\textcolor[rgb]{0.00,0.00,0.81}{#1}}
\newcommand{\FloatTok}[1]{\textcolor[rgb]{0.00,0.00,0.81}{#1}}
\newcommand{\ConstantTok}[1]{\textcolor[rgb]{0.00,0.00,0.00}{#1}}
\newcommand{\CharTok}[1]{\textcolor[rgb]{0.31,0.60,0.02}{#1}}
\newcommand{\SpecialCharTok}[1]{\textcolor[rgb]{0.00,0.00,0.00}{#1}}
\newcommand{\StringTok}[1]{\textcolor[rgb]{0.31,0.60,0.02}{#1}}
\newcommand{\VerbatimStringTok}[1]{\textcolor[rgb]{0.31,0.60,0.02}{#1}}
\newcommand{\SpecialStringTok}[1]{\textcolor[rgb]{0.31,0.60,0.02}{#1}}
\newcommand{\ImportTok}[1]{#1}
\newcommand{\CommentTok}[1]{\textcolor[rgb]{0.56,0.35,0.01}{\textit{#1}}}
\newcommand{\DocumentationTok}[1]{\textcolor[rgb]{0.56,0.35,0.01}{\textbf{\textit{#1}}}}
\newcommand{\AnnotationTok}[1]{\textcolor[rgb]{0.56,0.35,0.01}{\textbf{\textit{#1}}}}
\newcommand{\CommentVarTok}[1]{\textcolor[rgb]{0.56,0.35,0.01}{\textbf{\textit{#1}}}}
\newcommand{\OtherTok}[1]{\textcolor[rgb]{0.56,0.35,0.01}{#1}}
\newcommand{\FunctionTok}[1]{\textcolor[rgb]{0.00,0.00,0.00}{#1}}
\newcommand{\VariableTok}[1]{\textcolor[rgb]{0.00,0.00,0.00}{#1}}
\newcommand{\ControlFlowTok}[1]{\textcolor[rgb]{0.13,0.29,0.53}{\textbf{#1}}}
\newcommand{\OperatorTok}[1]{\textcolor[rgb]{0.81,0.36,0.00}{\textbf{#1}}}
\newcommand{\BuiltInTok}[1]{#1}
\newcommand{\ExtensionTok}[1]{#1}
\newcommand{\PreprocessorTok}[1]{\textcolor[rgb]{0.56,0.35,0.01}{\textit{#1}}}
\newcommand{\AttributeTok}[1]{\textcolor[rgb]{0.77,0.63,0.00}{#1}}
\newcommand{\RegionMarkerTok}[1]{#1}
\newcommand{\InformationTok}[1]{\textcolor[rgb]{0.56,0.35,0.01}{\textbf{\textit{#1}}}}
\newcommand{\WarningTok}[1]{\textcolor[rgb]{0.56,0.35,0.01}{\textbf{\textit{#1}}}}
\newcommand{\AlertTok}[1]{\textcolor[rgb]{0.94,0.16,0.16}{#1}}
\newcommand{\ErrorTok}[1]{\textcolor[rgb]{0.64,0.00,0.00}{\textbf{#1}}}
\newcommand{\NormalTok}[1]{#1}
\usepackage{longtable,booktabs}
\usepackage{graphicx,grffile}
\makeatletter
\def\maxwidth{\ifdim\Gin@nat@width>\linewidth\linewidth\else\Gin@nat@width\fi}
\def\maxheight{\ifdim\Gin@nat@height>\textheight\textheight\else\Gin@nat@height\fi}
\makeatother
% Scale images if necessary, so that they will not overflow the page
% margins by default, and it is still possible to overwrite the defaults
% using explicit options in \includegraphics[width, height, ...]{}
\setkeys{Gin}{width=\maxwidth,height=\maxheight,keepaspectratio}
\usepackage[normalem]{ulem}
% avoid problems with \sout in headers with hyperref:
\pdfstringdefDisableCommands{\renewcommand{\sout}{}}
\IfFileExists{parskip.sty}{%
\usepackage{parskip}
}{% else
\setlength{\parindent}{0pt}
\setlength{\parskip}{6pt plus 2pt minus 1pt}
}
\setlength{\emergencystretch}{3em}  % prevent overfull lines
\providecommand{\tightlist}{%
  \setlength{\itemsep}{0pt}\setlength{\parskip}{0pt}}
\setcounter{secnumdepth}{0}
% Redefines (sub)paragraphs to behave more like sections
\ifx\paragraph\undefined\else
\let\oldparagraph\paragraph
\renewcommand{\paragraph}[1]{\oldparagraph{#1}\mbox{}}
\fi
\ifx\subparagraph\undefined\else
\let\oldsubparagraph\subparagraph
\renewcommand{\subparagraph}[1]{\oldsubparagraph{#1}\mbox{}}
\fi

%%% Use protect on footnotes to avoid problems with footnotes in titles
\let\rmarkdownfootnote\footnote%
\def\footnote{\protect\rmarkdownfootnote}

%%% Change title format to be more compact
\usepackage{titling}

% Create subtitle command for use in maketitle
\newcommand{\subtitle}[1]{
  \posttitle{
    \begin{center}\large#1\end{center}
    }
}

\setlength{\droptitle}{-2em}
  \title{Markdown}
  \pretitle{\vspace{\droptitle}\centering\huge}
  \posttitle{\par}
  \author{}
  \preauthor{}\postauthor{}
  \date{}
  \predate{}\postdate{}

\usepackage[
  backend=biber,
  style=alphabetic,
  sorting=ynt,
  citestyle=authoryear
  ]{biblatex}
\addbibresource{../lit/bib.bib}

\usepackage[utf8]{inputenc}
\usepackage[spanish]{babel}

%%%% Frames
\ifxetex
    \makeatletter % undo the wrong changes made by mathspec
    \let\RequirePackage\original@RequirePackage
    \let\usepackage\RequirePackage
    \makeatother
\fi

\usepackage{xcolor}
\usepackage[tikz]{bclogo}
\usepackage[framemethod=tikz]{mdframed}
\usepackage{lipsum}
\usepackage[many]{tcolorbox}

\definecolor{bgblue}{RGB}{245,243,253}
\definecolor{ttblue}{RGB}{91,194,224}
\definecolor{llred}{RGB}{255,228,225}
\definecolor{bbblack}{RGB}{0,0,0}

\mdfdefinestyle{mystyle}{%
  rightline=true,
  innerleftmargin=10,
  innerrightmargin=10,
  outerlinewidth=3pt,
  topline=false,
  rightline=true,
  bottomline=false,
  skipabove=\topsep,
  skipbelow=\topsep
}

\newtcolorbox{curiosidad}[1][]{
  breakable,
  title=#1,
  colback=white,
  colbacktitle=white,
  coltitle=black,
  fonttitle=\bfseries,
  bottomrule=0pt,
  toprule=0pt,
  leftrule=3pt,
  rightrule=3pt,
  titlerule=0pt,
  arc=0pt,
  outer arc=0pt,
  colframe=black,
}

\newtcolorbox{nota}[1][]{
  breakable,
  freelance,
  title=#1,
  colback=white,
  colbacktitle=white,
  coltitle=black,
  fonttitle=\bfseries,
  bottomrule=0pt,
  boxrule=0pt,
  colframe=white,
  overlay unbroken and first={
  \draw[red!75!black,line width=3pt]
    ([xshift=5pt]frame.north west) -- 
    (frame.north west) -- 
    (frame.south west);
  \draw[red!75!black,line width=3pt]
    ([xshift=-5pt]frame.north east) -- 
    (frame.north east) -- 
    (frame.south east);
  },
  overlay unbroken app={
  \draw[red!75!black,line width=3pt,line cap=rect]
    (frame.south west) -- 
    ([xshift=5pt]frame.south west);
  \draw[red!75!black,line width=3pt,line cap=rect]
    (frame.south east) -- 
    ([xshift=-5pt]frame.south east);
  },
  overlay middle and last={
  \draw[red!75!black,line width=3pt]
    (frame.north west) -- 
    (frame.south west);
  \draw[red!75!black,line width=3pt]
    (frame.north east) -- 
    (frame.south east);
  },
  overlay last app={
  \draw[red!75!black,line width=3pt,line cap=rect]
    (frame.south west) --
    ([xshift=5pt]frame.south west);
  \draw[red!75!black,line width=3pt,line cap=rect]
    (frame.south east) --
    ([xshift=-5pt]frame.south east);
  },
}

\begin{document}


Estamos acostumbrados a editores del tipo \emph{what you see is what you
get}. \texttt{Markdown} permite escribir contenidos en \textbf{texto
plano} con una sintáxis para darle formato. Sobre esta sintáxis la
herramienta de software en Perl convierte el texto plano a HTML
\parencite{markdown}.

Tiene un alfabeto y símbolos que, una vez procesados, se ven de cierta
forma. A diferencia de un editor como Word en el que la versión de cada
computadora cambia la manera en la que se ven los documentos, con
\texttt{Markdown} esto no sucede.

Este lenguaje se está volviendo cada vez más común y, en particular,
páginas como GitHub y reddit lo utilizan para sus comentarios.

La curva de aprendizaje es mínima. En lo que se va memorizando el
alfabeto de markdown, lo más útil es tener una lista de fácil acceso con
los caracteres más comúnes y lo que hacen. A continuación, se
proporciona un listado de la sintáxis más común
\footnote{Basado en \textcite{markdowncheet1} y en \textcite{markdowncheet2}.}.

\subsection{Encabezados}\label{encabezados}

\begin{verbatim}
# Nivel 1
## Nivel 2
### Nivel 3
#### Nivel 4

... y asi
\end{verbatim}

\section{Nivel 1}\label{nivel-1}

\subsection{Nivel 2}\label{nivel-2}

\subsubsection{Nivel 3}\label{nivel-3}

\paragraph{Nivel 4}\label{nivel-4}

\section{Lineas horizontales}\label{lineas-horizontales}

Con tres o mas de los siguientes

\begin{verbatim}
---

Guiones

***
Asteriscos

___

Guiones bajos
\end{verbatim}

\begin{center}\rule{0.5\linewidth}{\linethickness}\end{center}

Guiones

\begin{center}\rule{0.5\linewidth}{\linethickness}\end{center}

Asteriscos

\begin{center}\rule{0.5\linewidth}{\linethickness}\end{center}

Guiones bajos

\section{Énfasis}\label{enfasis}

\begin{verbatim}
*italica* o _italica_
**negritas** o __negritas__
**_combinado_**
**Uno en negritas _el otro combinado_**
~~tachar~~
\end{verbatim}

\begin{itemize}
\tightlist
\item
  \emph{italica} o \emph{italica}
\item
  \textbf{negritas} o \textbf{negritas}
\item
  \textbf{\emph{combinado}}
\item
  \textbf{Uno en negritas \emph{el otro combinado}}
\item
  \sout{tachar}
\end{itemize}

\section{Bloques}\label{bloques}

\begin{verbatim}
> Ejemplo: Un bloque de varias
> lineas.
\end{verbatim}

\begin{quote}
Ejemplo: Un bloque de varias líneas.
\end{quote}

\section{Listas}\label{listas}

\begin{verbatim}
1. Primero
2. Segundo
    - Primer elemento del segundo
    - Segundo elemento del segundo
7. No tengo que cambiar el nombre, se pondra el correcto
    i. Una sublista con incisos
    ii. Mas incisos
4. Mas cosas
    a. Otra cosa
    b. Una mas
5. Otra manera de hacer listas no ordenadas
    * Usando asteriscos
    - O usando menos
    - O usando el signo de mas
\end{verbatim}

\begin{enumerate}
\def\labelenumi{\arabic{enumi}.}
\tightlist
\item
  Primero
\item
  Segundo

  \begin{itemize}
  \tightlist
  \item
    Primer elemento del segundo
  \item
    Segundo elemento del segundo
  \end{itemize}
\item
  No tengo que cambiar el nombre, se pondrá el correcto

  \begin{enumerate}
  \def\labelenumii{\roman{enumii}.}
  \tightlist
  \item
    Una sublista con incisos
  \item
    Mas incisos
  \end{enumerate}
\item
  Mas cosas

  \begin{enumerate}
  \def\labelenumii{\alph{enumii}.}
  \tightlist
  \item
    Otra cosa
  \item
    Una mas
  \end{enumerate}
\item
  Otra manera de hacer listas no ordenadas

  \begin{itemize}
  \tightlist
  \item
    Usando asteriscos
  \item
    O usando menos
  \item
    O usando el signo de mas
  \end{itemize}
\end{enumerate}

\section{Links}\label{links}

\begin{verbatim}
[Esto es lo que se ve](https://www.google.com)
[Esto es lo que se ve y el mensaje "Google" aparece en el hover]
(https://www.google.com "Google")
[Puedo hacer referencia a un archivo local](readme.md)
[Puedes poner referencias][1]

Detecata urls completos www.google.com o google.com

Y luego pones a donde te lleva la referencia
[1]: https://en.wikipedia.org/wiki/42_%28number%29#The_Hitchhiker.27s_Guide_to_the_Galaxy



\end{verbatim}

\href{https://www.google.com}{Esto es lo que se ve}

\href{https://www.google.com}{Esto es lo que se ve y el mensaje
``Google'' aparece en el hover}

\href{readme.md}{Puedo hacer referencia a un archivo local}

\href{https://en.wikipedia.org/wiki/42_\%28number\%29\#The_Hitchhiker.27s_Guide_to_the_Galaxy}{Puedes
poner referencias con numeros}

Luego quieres hacer referencias a otra parte del documento
\href{http://www.reddit.com}{por lo tanto, puedes ligar texto}

Detecta urls completos \url{http://www.google.com}

Y luego pones a dónde te lleva la referencia

Nota que los espacios entre lineas son importantes.

\section{Imágenes}\label{imagenes}

\begin{verbatim}
Puedes ponerlo en una misma linea: 
![alt text](dw.png "Es lo mejor")

Reference-style: 
![alt text][logo]

[logo]: dw.png "Sin duda alguna"
\end{verbatim}

Puedes ponerlo en una misma linea: \includegraphics{dw.png}

Reference-style: \includegraphics{dw.png}

\section{Tablas}\label{tablas}

\begin{verbatim}
Una tabla simple 

Encabezado 1  | Encabezado dos
------------- | --------------
Contenido     | Contenido
Contenido     | Contenido 
\end{verbatim}

\begin{longtable}[]{@{}ll@{}}
\toprule
Encabezado 1 & Encabezado dos\tabularnewline
\midrule
\endhead
Contenido & Contenido\tabularnewline
Contenido & Contenido\tabularnewline
\bottomrule
\end{longtable}

\begin{verbatim}
Una tabla alineada. La alineacion la puedes hacer por columnas

| Derecha | Centro | Iquierda |
| ----: | :----: | :---- |
| 10    | 10     | 10    |
| 1000  | 1000   | 1000  |
\end{verbatim}

\begin{longtable}[]{@{}rcl@{}}
\toprule
Derecha & Centro & Izquierda\tabularnewline
\midrule
\endhead
10 & 10 & 10\tabularnewline
1000 & 1000 & 1000\tabularnewline
\bottomrule
\end{longtable}

\section{HTML}\label{html}

Si sabes html, puedes utilizarlo directo.

\begin{verbatim}
<dl>
  <dt>Lista de definiciones</dt>
  <dd> Una def.</dd>

  <dt>Markdown en HTML</dt>
  <dd>No *siempre* funciona **bien**. Puro HTML <em>para que funcione</em>.</dd>
</dl>

Puedes controlar mejor las imagenes

<img src="img/imagen.png" align="middle" height="50" width="75" margin="0 auto" />
\end{verbatim}

Lista de definiciones

Una def.

Markdown en HTML

No \emph{siempre} funciona muy \textbf{bien}. Usa puro HTML para que
funcione siempre.

Puedes controlar mejor las imágenes

\section{Código}\label{codigo}

\begin{verbatim}
En una linea puedo poner `código`
\end{verbatim}

En una linea puedo poner \texttt{código}.

Para poner bloques de código se utilizan tres acentos invertidos y se
especifica el lenguaje.

\begin{Shaded}
\begin{Highlighting}[]
\ImportTok{import}\NormalTok{ requests}
\BuiltInTok{print} \StringTok{"¡Hola mundo!"}
\end{Highlighting}
\end{Shaded}

\begin{Shaded}
\begin{Highlighting}[]
\KeywordTok{select}\NormalTok{ * }\KeywordTok{from}\NormalTok{ tabla;}
\end{Highlighting}
\end{Shaded}

\begin{Shaded}
\begin{Highlighting}[]
\KeywordTok{library}\NormalTok{(dplyr)}
\end{Highlighting}
\end{Shaded}

\begin{verbatim}
Sin lenguaje, no lo resalta
\end{verbatim}

\section{Párrafos}\label{parrafos}

\begin{verbatim}
Si
escribo asi
me lo junta todo.

Debo separar para iniciar otro parrafo.
\end{verbatim}

Si escribo así me lo junta todo.

Debo separar para iniciar otro párrafo.


\end{document}
