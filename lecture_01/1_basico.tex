\documentclass[]{article}
\usepackage{lmodern}
\usepackage{amssymb,amsmath}
\usepackage{ifxetex,ifluatex}
\usepackage{fixltx2e} % provides \textsubscript
\ifnum 0\ifxetex 1\fi\ifluatex 1\fi=0 % if pdftex
  \usepackage[T1]{fontenc}
  \usepackage[utf8]{inputenc}
\else % if luatex or xelatex
  \ifxetex
    \usepackage{mathspec}
    \usepackage{xltxtra,xunicode}
  \else
    \usepackage{fontspec}
  \fi
  \defaultfontfeatures{Mapping=tex-text,Scale=MatchLowercase}
  \newcommand{\euro}{€}
\fi
% use upquote if available, for straight quotes in verbatim environments
\IfFileExists{upquote.sty}{\usepackage{upquote}}{}
% use microtype if available
\IfFileExists{microtype.sty}{%
\usepackage{microtype}
\UseMicrotypeSet[protrusion]{basicmath} % disable protrusion for tt fonts
}{}
\usepackage[margin=1in]{geometry}
\usepackage{color}
\usepackage{fancyvrb}
\newcommand{\VerbBar}{|}
\newcommand{\VERB}{\Verb[commandchars=\\\{\}]}
\DefineVerbatimEnvironment{Highlighting}{Verbatim}{commandchars=\\\{\}}
% Add ',fontsize=\small' for more characters per line
\usepackage{framed}
\definecolor{shadecolor}{RGB}{248,248,248}
\newenvironment{Shaded}{\begin{snugshade}}{\end{snugshade}}
\newcommand{\KeywordTok}[1]{\textcolor[rgb]{0.13,0.29,0.53}{\textbf{{#1}}}}
\newcommand{\DataTypeTok}[1]{\textcolor[rgb]{0.13,0.29,0.53}{{#1}}}
\newcommand{\DecValTok}[1]{\textcolor[rgb]{0.00,0.00,0.81}{{#1}}}
\newcommand{\BaseNTok}[1]{\textcolor[rgb]{0.00,0.00,0.81}{{#1}}}
\newcommand{\FloatTok}[1]{\textcolor[rgb]{0.00,0.00,0.81}{{#1}}}
\newcommand{\CharTok}[1]{\textcolor[rgb]{0.31,0.60,0.02}{{#1}}}
\newcommand{\StringTok}[1]{\textcolor[rgb]{0.31,0.60,0.02}{{#1}}}
\newcommand{\CommentTok}[1]{\textcolor[rgb]{0.56,0.35,0.01}{\textit{{#1}}}}
\newcommand{\OtherTok}[1]{\textcolor[rgb]{0.56,0.35,0.01}{{#1}}}
\newcommand{\AlertTok}[1]{\textcolor[rgb]{0.94,0.16,0.16}{{#1}}}
\newcommand{\FunctionTok}[1]{\textcolor[rgb]{0.00,0.00,0.00}{{#1}}}
\newcommand{\RegionMarkerTok}[1]{{#1}}
\newcommand{\ErrorTok}[1]{\textbf{{#1}}}
\newcommand{\NormalTok}[1]{{#1}}
\ifxetex
  \usepackage[setpagesize=false, % page size defined by xetex
              unicode=false, % unicode breaks when used with xetex
              xetex]{hyperref}
\else
  \usepackage[unicode=true]{hyperref}
\fi
\hypersetup{breaklinks=true,
            bookmarks=true,
            pdfauthor={},
            pdftitle={R: lo básico},
            colorlinks=true,
            citecolor=blue,
            urlcolor=blue,
            linkcolor=magenta,
            pdfborder={0 0 0}}
\urlstyle{same}  % don't use monospace font for urls
\setlength{\parindent}{0pt}
\setlength{\parskip}{6pt plus 2pt minus 1pt}
\setlength{\emergencystretch}{3em}  % prevent overfull lines
\setcounter{secnumdepth}{0}

%%% Use protect on footnotes to avoid problems with footnotes in titles
\let\rmarkdownfootnote\footnote%
\def\footnote{\protect\rmarkdownfootnote}

%%% Change title format to be more compact
\usepackage{titling}

% Create subtitle command for use in maketitle
\newcommand{\subtitle}[1]{
  \posttitle{
    \begin{center}\large#1\end{center}
    }
}

\setlength{\droptitle}{-2em}
  \title{R: lo básico}
  \pretitle{\vspace{\droptitle}\centering\huge}
  \posttitle{\par}
  \author{}
  \preauthor{}\postauthor{}
  \date{}
  \predate{}\postdate{}

\usepackage[
  backend=biber,
  style=alphabetic,
  sorting=ynt,
  citestyle=authoryear
  ]{biblatex}
\addbibresource{../lit/bib.bib}

\usepackage[utf8]{inputenc}
\usepackage[spanish]{babel}


\begin{document}

\maketitle


\section{El espacio de trabajo
(Workspace)}\label{el-espacio-de-trabajo-workspace}

\subsection{Directorio de trabajo}\label{directorio-de-trabajo}

El directorio de trabajo o \emph{working directory} es el folder en tu
computadora en el que estás trabajando en ese momento. Cuando se le pide
a R que abra un archivo o guarde ciertos datos, R lo hará a partir del
directorio de trabajo que le hayas fijado.

Para saber en qué directorio te encuentras, se usa el comando
\texttt{getwd()}.

\begin{Shaded}
\begin{Highlighting}[]
\KeywordTok{getwd}\NormalTok{()}
\end{Highlighting}
\end{Shaded}

\begin{verbatim}
## [1] "/home/animalito/study/aprendeR/lecture_01"
\end{verbatim}

Para especificar el directorio de trabajo, se utiliza el comando
\texttt{setwd()} en la consola. Y volvemos a

\begin{Shaded}
\begin{Highlighting}[]
\KeywordTok{setwd}\NormalTok{(}\StringTok{"/home/animalito/study/"}\NormalTok{)}
\KeywordTok{getwd}\NormalTok{()}
\end{Highlighting}
\end{Shaded}

Con lo que acabamos de hacer, R buscará archivos o guardará archivos en
el folder \texttt{/home/animalito/study/}. En R también es posible
navegar a partir de el directorio de trabajo. Como siempre,

\begin{itemize}
\itemsep1pt\parskip0pt\parsep0pt
\item
  ``../un\_archivo.R'' le indica a R que busque un folder arriva del
  actual directorio de trabajo por el archivo \emph{un\_archivo.R}.
\item
  ``datos/otro\_archivo.R'' hace que se busque en el directorio de
  trabajo, dentro del folder \emph{datos} por el archivo
  \emph{otro\_archivo.R}
\end{itemize}

\subsection{Ejemplos básicos}\label{ejemplos-basicos}

La consola permite hacer operaciones sobre números o caracteres (cuando
tiene sentido).

\begin{Shaded}
\begin{Highlighting}[]
\CommentTok{# Potencias, sumas, multiplicaciones}
\DecValTok{2}\NormalTok{^}\DecValTok{3} \NormalTok{+}\StringTok{ }\DecValTok{67} \NormalTok{*}\StringTok{ }\DecValTok{4} \NormalTok{-}\StringTok{ }\NormalTok{(}\DecValTok{45} \NormalTok{+}\StringTok{ }\DecValTok{5}\NormalTok{)}
\end{Highlighting}
\end{Shaded}

\begin{verbatim}
## [1] 226
\end{verbatim}

\begin{Shaded}
\begin{Highlighting}[]
\CommentTok{# Comparaciones}
\DecValTok{56} \NormalTok{>}\StringTok{ }\DecValTok{78} 
\end{Highlighting}
\end{Shaded}

\begin{verbatim}
## [1] FALSE
\end{verbatim}

\begin{Shaded}
\begin{Highlighting}[]
\DecValTok{34} \NormalTok{<=}\StringTok{ }\DecValTok{34}
\end{Highlighting}
\end{Shaded}

\begin{verbatim}
## [1] TRUE
\end{verbatim}

\begin{Shaded}
\begin{Highlighting}[]
\DecValTok{234} \NormalTok{<}\StringTok{ }\DecValTok{345}
\end{Highlighting}
\end{Shaded}

\begin{verbatim}
## [1] TRUE
\end{verbatim}

\begin{Shaded}
\begin{Highlighting}[]
\StringTok{"hola"} \NormalTok{==}\StringTok{ "hola"}
\end{Highlighting}
\end{Shaded}

\begin{verbatim}
## [1] TRUE
\end{verbatim}

\begin{Shaded}
\begin{Highlighting}[]
\CommentTok{# modulo}
\DecValTok{10} \NormalTok\StringTok{ }\DecValTok{4} 
\end{Highlighting}
\end{Shaded}

\begin{verbatim}
## [1] 2
\end{verbatim}

Estas operaciones también pueden ser realizadas entre vectores

\begin{Shaded}
\begin{Highlighting}[]
\NormalTok{x <-}\StringTok{ }\NormalTok{-}\DecValTok{1}\NormalTok{:}\DecValTok{12}
\NormalTok{x}
\end{Highlighting}
\end{Shaded}

\begin{verbatim}
##  [1] -1  0  1  2  3  4  5  6  7  8  9 10 11 12
\end{verbatim}

\begin{Shaded}
\begin{Highlighting}[]
\NormalTok{x +}\StringTok{ }\DecValTok{1}
\end{Highlighting}
\end{Shaded}

\begin{verbatim}
##  [1]  0  1  2  3  4  5  6  7  8  9 10 11 12 13
\end{verbatim}

\begin{Shaded}
\begin{Highlighting}[]
\DecValTok{2} \NormalTok{*}\StringTok{ }\NormalTok{x +}\StringTok{ }\DecValTok{3}
\end{Highlighting}
\end{Shaded}

\begin{verbatim}
##  [1]  1  3  5  7  9 11 13 15 17 19 21 23 25 27
\end{verbatim}

\begin{Shaded}
\begin{Highlighting}[]
\NormalTok{x %%}\StringTok{ }\DecValTok{5} \CommentTok{#-- is periodic}
\end{Highlighting}
\end{Shaded}

\begin{verbatim}
##  [1] 4 0 1 2 3 4 0 1 2 3 4 0 1 2
\end{verbatim}

\begin{Shaded}
\begin{Highlighting}[]
\NormalTok{x %/%}\StringTok{ }\DecValTok{5}
\end{Highlighting}
\end{Shaded}

\begin{verbatim}
##  [1] -1  0  0  0  0  0  1  1  1  1  1  2  2  2
\end{verbatim}

\subsection{Comandos útiles}\label{comandos-utiles}

Para enlistar lso objetos que están en el espacio de trabajo

\begin{Shaded}
\begin{Highlighting}[]
\KeywordTok{ls}\NormalTok{()}
\end{Highlighting}
\end{Shaded}

\begin{verbatim}
## [1] "x"
\end{verbatim}

Para eliminar todos los objetos en un workspace

\begin{Shaded}
\begin{Highlighting}[]
\NormalTok{rm(list = ls(}\ErrorTok{))} \CommentTok{# se puede borrar solo uno, por ejemplo, nombrándolo}
\KeywordTok{ls}\NormalTok{()}
\end{Highlighting}
\end{Shaded}

\begin{verbatim}
## character(0)
\end{verbatim}

También se puede utilizar/guardar la historia de comandos utilizados

\begin{Shaded}
\begin{Highlighting}[]
\KeywordTok{history}\NormalTok{()}
\KeywordTok{history}\NormalTok{(}\DataTypeTok{max.show =} \DecValTok{5}\NormalTok{)}
\KeywordTok{history}\NormalTok{(}\DataTypeTok{max.show =} \OtherTok{Inf}\NormalTok{) }\CommentTok{# Muestra toda la historia}

\CommentTok{# Se puede salvar la historia de comandos a un archivo}
\KeywordTok{savehistory}\NormalTok{(}\DataTypeTok{file =} \StringTok{"mihistoria"}\NormalTok{) }\CommentTok{# Por default, R ya hace esto }
\CommentTok{# en un archivo ".Rhistory"}

\CommentTok{# Cargar al current workspace una historia de comandos en particular}
\KeywordTok{loadhistory}\NormalTok{(}\DataTypeTok{file =} \StringTok{"mihistoria"}\NormalTok{)}
\end{Highlighting}
\end{Shaded}

Es posible también guardar el workspace -en forma completa- en un
archivo con el comando \texttt{save.image()} a un archivo con extensión
\emph{.RData}. Puedes guardar una lista de objetos específica a un
archivo \emph{.RData}. Por ejemplo:

\begin{Shaded}
\begin{Highlighting}[]
\NormalTok{x <-}\StringTok{ }\DecValTok{1}\NormalTok{:}\DecValTok{12}
\NormalTok{y <-}\StringTok{ }\DecValTok{3}\NormalTok{:}\DecValTok{45}
\NormalTok{save(x, y, file = }\StringTok{"ejemplo.RData"}\ErrorTok{)} \CommentTok{#la extensión puede ser arbitraria.}
\end{Highlighting}
\end{Shaded}

Después puedo cargar ese archivo. Prueba hacer:

\begin{Shaded}
\begin{Highlighting}[]
\KeywordTok{rm}\NormalTok{(}\DataTypeTok{list =} \KeywordTok{ls}\NormalTok{()) }\CommentTok{# limpiamos workspace}
\NormalTok{load(file = }\StringTok{"ejemplo.RData"}\ErrorTok{)} \CommentTok{#la extensión puede ser arbitraria.}
\KeywordTok{ls}\NormalTok{()}
\end{Highlighting}
\end{Shaded}

Nota como los objetos preservan el nombre con el que fueron guardados.

\section{Librerías}\label{librerias}

R puede hacer muchos análisis estadísticos y de datos. Las diferentes
capacidades están organizadas en paquetes o librerías. Con la
\href{https://github.com/animalito/aprendeR/blob/master/lecture_01/0_instalacion.pdf}{instalación
estándar} se instalan también las librerías más comunes. Para obtener
una lista de todos los paquetes instalados se puede utilizar el comando
\texttt{library()} en la consola.

Existen una gran cantidad de paquetes disponibles además de los
incluidos por default.

\subsection{CRAN}\label{cran}

CRAN o \emph{Comprehensive R Archive Network} es una colección de sitios
que contienen exactamente el mismo material, es decir, las
distribuciones de R, las extensiones, la documentación y los binarios.
El master de CRAN está en Wirtschaftsuniversität Wien en Austria. Éste
se ``espeja'' (\emph{mirrors}) en forma diaria a muchos sitios alrededor
del mundo. En la \href{https://cran.r-project.org/mirrors.html}{lista de
espejos} se puede ver que para México están disponibles el espejo del
ITAM, del Colegio de Postgraduados (Texcoco) y Jellyfish Foundation.

Los espejos son importantes pues, cada vez que busquen instalar
paquetes, se les preguntará qué espejo quieren utilizar para la sesión
en cuestión. Del espejo que selecciones, será del cuál R \emph{bajará}
el binario y la documentación.

Del CRAN es que se obtiene la última versión oficial de R. Diario se
actualizan los espejos. Para más detalles consultar el
\href{https://cran.r-project.org/doc/FAQ/R-FAQ.html}{FAQ}.

Para contribuir un paquete en CRAN se deben seguir las instrucciones
\href{https://cran.r-project.org/web/packages/policies.html}{aquí}.

\subsection{Github}\label{github}

Git es un controlador de versiones muy popular para desarrollar
software. Cuando se combina con \href{https://github.com/}{GitHub} se
puede compartir el código con el resto de la comunidad. Éste controlador
de versiones es el más popular entre los que contribuyen a R. Muchos
problemas a los que uno se enfrenta alguien ya los desarrolló y no
necesariamente publicó el paquete en CRAN. Para instalar algún paquete
desde GitHub, se pueden seguir las instrucciones siguientes

\begin{Shaded}
\begin{Highlighting}[]
\KeywordTok{install.packages}\NormalTok{(}\StringTok{"devtools"}\NormalTok{)}
\NormalTok{devtools::}\KeywordTok{install_github}\NormalTok{(}\StringTok{"username/packagename"}\NormalTok{)}
\end{Highlighting}
\end{Shaded}

Donde \texttt{username} es el usuario de Github y \texttt{packagename}
es el nombre del repositorio que contiene el paquete. Cuidado, no todo
repositorio en GitHub es un paquete. Para más información ver el
capítulo \href{http://r-pkgs.had.co.nz/git.html}{Git and GitHub} en
\textcite{wickham2015r}.

\subsection{Otras fuentes}\label{otras-fuentes}

Otros lugares en donde es común que se publiquen paquetes es en
\href{https://www.bioconductor.org/}{Bioconductor} un projecto de
software para la comprensión de datos del genoma humano.

\section{Paquetes recomendados}\label{paquetes-recomendados}

Hay muchísimas librerías y lo recomendable es, dado un problema y un
modelo para resolverlo, revisar si alguien ya implementó el método en
algunas de las fuentes de paquetes mencionadas antes. Para una lista de
paquetes que son de mucha utilidad ver
\href{https://github.com/Skalas/massive-adventure-ubuntu/blob/master/Rpackages/data_manipulation.R}{estas
recomendaciones}.

\section{Scripting}\label{scripting}

R es un intérprete. Utiliza un ambiente basado en línea de comandos. Por
ende, es necesario escribir la secuencia de comandos que se desea
realizar a diferencia de otras herramientas en donde es posible utlizar
el mouse o menús.

Aunque los comandos pueden ser ejecutados directamente en consola una
única vez, también es posible guardarlos en archivos conocidos como
\emph{scripts}. Típicamente, utilizamos la extensión \textbf{.R} o
\textbf{.r}. En RStudio, \texttt{CTRL\ +\ SHIFT\ +\ N} abre
inmediatamente un nuevo editor en el panel superior izquierdo.

Se puede \emph{ir editando} el script y corriendo los comandos línea por
línea con \texttt{CTRL\ +\ ENTER}. Esto también aplica para
\emph{correr} una selección del texto editable.

Es posible también correr todo el script

\begin{Shaded}
\begin{Highlighting}[]
\KeywordTok{source}\NormalTok{(}\StringTok{"foo.R"}\NormalTok{)}
\end{Highlighting}
\end{Shaded}

O con el atajo \texttt{CTRL\ +\ SHIFT\ +\ S} en RStudio.

Para enlistar algunos shortcuts comunes en RStudio presiona
\texttt{ALT\ +\ SHIFT\ +\ K}. De la misma manera, si utilizas Emacs +
ESS, existen múltiples atajos de teclado para realizar todo mucho más
eficientemente. Estudiarlos no es tiempo perdido.

\section{Ayuda \& documentación}\label{ayuda-documentacion}

R tiene mucha documentación. Desde la consola se puede accesar a la
misma.

Para ayuda general,

\begin{Shaded}
\begin{Highlighting}[]
\KeywordTok{help.start}\NormalTok{()}
\end{Highlighting}
\end{Shaded}

Para la ayuda de una función en especifico, por ejemplo, si se quiere
graficar algo y sabemos que existe la funcion \texttt{plot} podemos
consultar fácilmente la ayuda.

\begin{Shaded}
\begin{Highlighting}[]
\KeywordTok{help}\NormalTok{(plot)}
\CommentTok{# o tecleando directamente}
\NormalTok{?plot}
\end{Highlighting}
\end{Shaded}

El segundo ejemplo se puede extender para buscar esa función en todos
los paquetes que tengo instalados en mi ambiente al escribir
\texttt{??plot}.

La documetnación normalmente se acompaña de ejemplos. Para \emph{correr}
los ejemplos sin necesidad de copiar y pegar, prueba

\begin{Shaded}
\begin{Highlighting}[]
\KeywordTok{example}\NormalTok{(plot)}
\end{Highlighting}
\end{Shaded}

Para búsquedas más comprensivas, se puede buscar de otras maneras:

\begin{Shaded}
\begin{Highlighting}[]
\KeywordTok{apropos}\NormalTok{(}\StringTok{"foo"}\NormalTok{) }\CommentTok{# Enlista todas las funciones que contengan la cadena "foo"}
\KeywordTok{RSiteSearch}\NormalTok{(}\StringTok{"foo"}\NormalTok{) }\CommentTok{# Busca por la cadena "foo" en todos los manuales de ayuda }
\CommentTok{# y listas de distribución.}
\end{Highlighting}
\end{Shaded}

\section{Estructuras de datos}\label{estructuras-de-datos}

\subsection{Vectores}\label{vectores}

\subsection{Matrices}\label{matrices}

\subsection{Data frames}\label{data-frames}

\subsection{Listas}\label{listas}

\section{Leer y escribir archivos de
datos}\label{leer-y-escribir-archivos-de-datos}

\subsection{Lectura}\label{lectura}

Desde archivo

Datos de muestra

\subsection{Escritura}\label{escritura}

\section{Estructuras de programación}\label{estructuras-de-programacion}

\subsection{IFs}\label{ifs}

\subsection{Fors}\label{fors}

\subsection{Whiles}\label{whiles}

\subsection{Funciones propias}\label{funciones-propias}

\printbibliography

\end{document}
