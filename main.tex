\documentclass[12pt]{book}
\usepackage[
backend=biber,
style=alphabetic,
sorting=ynt,
citestyle=authoryear
]{biblatex}
\addbibresource{Bibliography.bib}
\usepackage{csquotes}
\usepackage{enumitem}
\usepackage{verbatim}  % Needed for the "comment" environment to make LaTeX comments
\usepackage{vector}  % Allows "\bvec{}" and "\buvec{}" for "blackboard" style bold vectors in maths
\hypersetup{urlcolor=blue, colorlinks=true}  % Colours hyperlinks in blue, but this can be distracting if there are many links.

\usepackage[utf8]{inputenc}
\usepackage[spanish]{babel}
\usepackage{longtable}

\usepackage{pdfpages}
\usepackage{listings}
\usepackage{float}
%% USA 12.5 CENTIMETROS PARA EL MARGEN!
%% ----------------------------------------------------------------
\begin{document}
\frontmatter      % Begin Roman style (i, ii, iii, iv...) page numbering

% Set up the Title Page
\title  {Thesis Title}
\authors  {\texorpdfstring
            {\href{animalsalvaje@gmail.com}{Andrea Fern\'andez Conde}}
            {\href{escalas5@gmail.com}{Miguel Angel Escalante Serrato}}
            }
\addresses  {\groupname\\\deptname\\\univname}  % Do not change this here, instead these must be set in the "Thesis.cls" file, please look through it instead
\date       {\today}
\subject    {}
\keywords   {}
% % title page commented for now
% \maketitle
%% ----------------------------------------------------------------

\setstretch{1.3}  % It is better to have smaller font and larger line spacing than the other way round

% Define the page headers using the FancyHdr package and set up for one-sided printing
\fancyhead{}  % Clears all page headers and footers
\rhead{\thepage}  % Sets the right side header to show the page number
\lhead{}  % Clears the left side page header

\pagestyle{fancy}  % Finally, use the "fancy" page style to implement the FancyHdr headers

%% ----------------------------------------------------------------
%% % Declaration Page required for the Thesis, your institution may give you a different text to place here
%% \Declaration{

%% \addtocontents{toc}{\vspace{1em}}  % Add a gap in the Contents, for aesthetics

%% I, AUTHOR NAME, declare that this thesis titled, `THESIS TITLE' and the work presented in it are my own. I confirm that:

%% \begin{itemize} 
%% \item[\tiny{$\blacksquare$}] This work was done wholly or mainly while in candidature for a research degree at this University.
 
%% \item[\tiny{$\blacksquare$}] Where any part of this thesis has previously been submitted for a degree or any other qualification at this University or any other institution, this has been clearly stated.
 
%% \item[\tiny{$\blacksquare$}] Where I have consulted the published work of others, this is always clearly attributed.
 
%% \item[\tiny{$\blacksquare$}] Where I have quoted from the work of others, the source is always given. With the exception of such quotations, this thesis is entirely my own work.
 
%% \item[\tiny{$\blacksquare$}] I have acknowledged all main sources of help.
 
%% \item[\tiny{$\blacksquare$}] Where the thesis is based on work done by myself jointly with others, I have made clear exactly what was done by others and what I have contributed myself.
%% \\
%% \end{itemize}
 
 
%% Signed:\\
%% \rule[1em]{25em}{0.5pt}  % This prints a line for the signature
 
%% Date:\\
%% \rule[1em]{25em}{0.5pt}  % This prints a line to write the date
%% }
%% \clearpage  % Declaration ended, now start a new page

%% ----------------------------------------------------------------
%% % The "Funny Quote Page"
%% \pagestyle{empty}  % No headers or footers for the following pages

%% \null\vfill
%% % Now comes the "Funny Quote", written in italics
%% \textit{``Write a funny quote here.''}

%% \begin{flushright}
%% If the quote is taken from someone, their name goes here
%% \end{flushright}

%% \vfill\vfill\vfill\vfill\vfill\vfill\null
%% \clearpage  % Funny Quote page ended, start a new page
%% ----------------------------------------------------------------

%% % The Abstract Page
%% \addtotoc{Resumen}  % Add the "Abstract" page entry to the Contents
%% \abstract{
%% \addtocontents{toc}{\vspace{1em}}  % Add a gap in the Contents, for aesthetics

%% The Thesis Abstract is written here (and usually kept to just this page). The page is kept centered vertically so can expand into the blank space above the title too\ldots

%% }

%% \clearpage  % Abstract ended, start a new page
%% ----------------------------------------------------------------

%% \setstretch{1.3}  % Reset the line-spacing to 1.3 for body text (if it has changed)

%% % The Acknowledgements page, for thanking everyone
%% \acknowledgements{
%% \addtocontents{toc}{\vspace{1em}}  % Add a gap in the Contents, for aesthetics

%% The acknowledgements and the people to thank go here, don't forget to include your project advisor\ldots

%% }
%% \clearpage  % End of the Acknowledgements
%% ----------------------------------------------------------------

\pagestyle{fancy}  %The page style headers have been "empty" all this time, now use the "fancy" headers as defined before to bring them back


%% ----------------------------------------------------------------
\lhead{\emph{\'Indice}}  % Set the left side page header to "Contents"
\tableofcontents  % Write out the Table of Contents

%% ----------------------------------------------------------------
% \lhead{\emph{Figuras}}  % Set the left side page header to "List if Figures"
% \listoffigures  % Write out the List of Figures

%% ----------------------------------------------------------------
% \lhead{\emph{Lista de Tablas}}  % Set the left side page header to "List of Tables"
% \listoftables  % Write out the List of Tables

%% ----------------------------------------------------------------
\setstretch{1.5}  % Set the line spacing to 1.5, this makes the following tables easier to read
\clearpage  % Start a new page
\lhead{\emph{Glosario}}  % Set the left side page header to "Abbreviations"
%\listofsymbols{ll}{}  % Include a list of Abbreviations (a table of two columns)
\chapter{Glosario}
\begin{center}
    \begin{longtable}{p{3cm}p{11.5cm}}
% \textbf{Acronym} & \textbf{W}hat (it) \textbf{S}tands \textbf{F}or \\
\textbf{INAI} & Instituto Nacional de Transparencia, Acceso a la Información y Protección de Datos Personales. \\
\textbf{DAI} & Derecho de Acceso a la Información. \\
\textbf{SNT} & Sistema Nacional de Transparencia, Acceso a la Información y Protección de Datos Personales. \\
\textbf{PNT} & Plataforma Nacional de Transparencia \\
\textbf{DGPA} & Dirección General de Políticas de Acceso. \\
\textbf{Documentación DGPA} & Incluye todos los metadatos, catálogos, documentos metodológicos, comentarios a códigos e instrucciones de reproducción. \\
\textbf{INFOMEX} & Sistema del Gobierno Federal para gestionar solicitudes de información.\\
\textbf{POT} & Portal de Obligaciones de Transparencia.\\
\textbf{H-COM} & Herramienta de comunicación interna del INAI.\\
\textbf{ZOOM} & Buscador de Solicitudes de Información y Recursos de Revisión. Es una herramienta de búsqueda de solicitudes que se han formulado al Gobierno Federal, de las respuestas que se han proporcionado, y de las resoluciones que el INAI emite ante las inconformidades de los ciudadanos respecto a las respuestas que obtienen. \\
\textbf{Diagnóstico \#MapaDAI} & Es un diagnóstico que integra la información de distintos subsistemas informáticos y bases de datos del INAI, así como de todos los demás órganos garantes a nivel nacional. \\
\textbf{Herramienta \#MapaDAI} & Es la herramienta que permite visualizar los resultados arrojados por el Diagnóstico \textbf{ \#MapaDAI}. \\
\textbf{DWH} & Un data warehouse o almacén de datos es una colección de datos orientada a un ámbito específico, en este caso, el ejercicio del DAI, que ayuda a la toma de decisiones en la Organización que la utiliza. Su diseño debe de favorecer el análisis y la divulgación de los datos. \\
\textbf{Data mart} & Es una versión particular de un DWH que tiene como objetivo satisfacer las necesidades de datos de un área específica, está orientado a la consulta de datos.\\
\textbf{ETL} & Extract, transform and load (extraer, transformar y cargar) es el proceso que permite acomodar datos provenientes de múltiples fuentes, hacer todas las transformaciones necesarias para cargarlos en otra base de datos, data mart o data warehouse.\\
\textbf{ORM} & Object-relational mapping (mapeo objeto relacional) es una técnica de programación para pasar de un sistema de tipos en programación orientada a objetos a una base de datos relacional.\\
\textbf{API} & Application Programming Interface (interfaz de programación de aplicaciones) incluye a todas las funciones que ofrece una biblioteca para ser utilizado por otro software. \\
\textbf{ODBC} & Open DataBase Connectivity es un estándar de acceso a una base de datos.\\
\textbf{Server} & Un servidor es una aplicación en ejecución que atiende las peticiones de un cliente y devuelve una respuesta.\\
\textbf{Host} & Un host es una computadora o dispositivo conectado a una red de computadoras. Éste puede ofrecer información, servicios o aplicaciones a otros nodos de la red. \\
\textbf{AWS} & Amazon Web Services es una colección de servicios de web en la nube ofrecidas por Amazon. \\
\textbf{Arquitectura} & A grandes rasgos, se entiende como el plan para integrar, centralizar y mantener los datos de distintas fuentes. \\
\textbf{Tecnologías} & Por tecnologías se entederá únicamente el software a utilizar.\\
\textbf{Flujo de datos} & Se incluye la ingesta de datos, su almacenamiento y los ETL's.\\
\textbf{Script} & Es un archivo de comandos secuenciales, generalmente utilizado en lenguajes interpretados. \\
\textbf{Dueño/administrador} & encagado(s) de los flujos de datos, arquitectura y tecnologías en cada subsistema.\\
\textbf{Protocolos de integración} & se entenderán por éstos los acuerdos con los dueños/administradores de los sistemas para utilizar los ETL's y el DWH construido, conectarlo a su flujo de datos y darle mantenimiento.\\
\textbf{Data wrangling o munging} & Término informal para denotar al proceso manual de convertir o mapear los datos desde un formato bruto (tal cuál se capturan) a uno más conveniente para el consumo de los datos. Se incluye en este término proceso adicionales a los datos como: transformaciones adicionales a estructuras de datos definidas, generación de variables, agregaciones, ordenamiento, entrenamiento de modelos estadísticos, almacenamiento, entre otros. \\
\end{longtable}
\end{center}

%% ----------------------------------------------------------------
% \clearpage  %Start a new page
% \lhead{\emph{Symbols}}  % Set the left side page header to "Symbols"
% \listofnomenclature{lll}  % Include a list of Symbols (a three column table)
% {
% % symbol & name & unit \\
% $a$ & distance & m \\
% $P$ & power & W (Js$^{-1}$) \\
% & & \\ % Gap to separate the Roman symbols from the Greek
% $\omega$ & angular frequency & rads$^{-1}$ \\
% }
%% ----------------------------------------------------------------
% End of the pre-able, contents and lists of things
% Begin the Dedication page

\setstretch{1.3}  % Return the line spacing back to 1.3

\pagestyle{empty}  % Page style needs to be empty for this page
% \dedicatory{For/Dedicated to/To my\ldots}

\addtocontents{toc}{\vspace{2em}}  % Add a gap in the Contents, for aesthetics


%% ----------------------------------------------------------------
\mainmatter	  % Begin normal, numeric (1,2,3...) page numbering
\pagestyle{fancy}  % Return the page headers back to the "fancy" style

% Include the chapters of the thesis, as separate files
% Just uncomment the lines as you write the chapters

\documentclass[]{article}
\usepackage{lmodern}
\usepackage{amssymb,amsmath}
\usepackage{ifxetex,ifluatex}
\usepackage{fixltx2e} % provides \textsubscript
\ifnum 0\ifxetex 1\fi\ifluatex 1\fi=0 % if pdftex
  \usepackage[T1]{fontenc}
  \usepackage[utf8]{inputenc}
\else % if luatex or xelatex
  \ifxetex
    \usepackage{mathspec}
  \else
    \usepackage{fontspec}
  \fi
  \defaultfontfeatures{Ligatures=TeX,Scale=MatchLowercase}
\fi
% use upquote if available, for straight quotes in verbatim environments
\IfFileExists{upquote.sty}{\usepackage{upquote}}{}
% use microtype if available
\IfFileExists{microtype.sty}{%
\usepackage{microtype}
\UseMicrotypeSet[protrusion]{basicmath} % disable protrusion for tt fonts
}{}
\usepackage[margin=1in]{geometry}
\usepackage{hyperref}
\hypersetup{unicode=true,
            pdftitle={Introducción},
            pdfborder={0 0 0},
            breaklinks=true}
\urlstyle{same}  % don't use monospace font for urls
\usepackage{graphicx,grffile}
\makeatletter
\def\maxwidth{\ifdim\Gin@nat@width>\linewidth\linewidth\else\Gin@nat@width\fi}
\def\maxheight{\ifdim\Gin@nat@height>\textheight\textheight\else\Gin@nat@height\fi}
\makeatother
% Scale images if necessary, so that they will not overflow the page
% margins by default, and it is still possible to overwrite the defaults
% using explicit options in \includegraphics[width, height, ...]{}
\setkeys{Gin}{width=\maxwidth,height=\maxheight,keepaspectratio}
\IfFileExists{parskip.sty}{%
\usepackage{parskip}
}{% else
\setlength{\parindent}{0pt}
\setlength{\parskip}{6pt plus 2pt minus 1pt}
}
\setlength{\emergencystretch}{3em}  % prevent overfull lines
\providecommand{\tightlist}{%
  \setlength{\itemsep}{0pt}\setlength{\parskip}{0pt}}
\setcounter{secnumdepth}{5}
% Redefines (sub)paragraphs to behave more like sections
\ifx\paragraph\undefined\else
\let\oldparagraph\paragraph
\renewcommand{\paragraph}[1]{\oldparagraph{#1}\mbox{}}
\fi
\ifx\subparagraph\undefined\else
\let\oldsubparagraph\subparagraph
\renewcommand{\subparagraph}[1]{\oldsubparagraph{#1}\mbox{}}
\fi

%%% Use protect on footnotes to avoid problems with footnotes in titles
\let\rmarkdownfootnote\footnote%
\def\footnote{\protect\rmarkdownfootnote}

%%% Change title format to be more compact
\usepackage{titling}

% Create subtitle command for use in maketitle
\newcommand{\subtitle}[1]{
  \posttitle{
    \begin{center}\large#1\end{center}
    }
}

\setlength{\droptitle}{-2em}
  \title{Introducción}
  \pretitle{\vspace{\droptitle}\centering\huge}
  \posttitle{\par}
  \author{}
  \preauthor{}\postauthor{}
  \date{}
  \predate{}\postdate{}

\usepackage[
  backend=biber,
  style=alphabetic,
  sorting=ynt,
  citestyle=authoryear
  ]{biblatex}
\addbibresource{../lit/bib.bib}

\usepackage[utf8]{inputenc}
\usepackage[spanish]{babel}

%%%% Frames
\ifxetex
    \makeatletter % undo the wrong changes made by mathspec
    \let\RequirePackage\original@RequirePackage
    \let\usepackage\RequirePackage
    \makeatother
\fi

\usepackage{xcolor}
\usepackage[tikz]{bclogo}
\usepackage[framemethod=tikz]{mdframed}
\usepackage{lipsum}
\usepackage[many]{tcolorbox}

\definecolor{bgblue}{RGB}{245,243,253}
\definecolor{ttblue}{RGB}{91,194,224}
\definecolor{llred}{RGB}{255,228,225}
\definecolor{bbblack}{RGB}{0,0,0}

\mdfdefinestyle{mystyle}{%
  rightline=true,
  innerleftmargin=10,
  innerrightmargin=10,
  outerlinewidth=3pt,
  topline=false,
  rightline=true,
  bottomline=false,
  skipabove=\topsep,
  skipbelow=\topsep
}

\newtcolorbox{curiosidad}[1][]{
  breakable,
  title=#1,
  colback=white,
  colbacktitle=white,
  coltitle=black,
  fonttitle=\bfseries,
  bottomrule=0pt,
  toprule=0pt,
  leftrule=3pt,
  rightrule=3pt,
  titlerule=0pt,
  arc=0pt,
  outer arc=0pt,
  colframe=black,
}

\newtcolorbox{nota}[1][]{
  breakable,
  freelance,
  title=#1,
  colback=white,
  colbacktitle=white,
  coltitle=black,
  fonttitle=\bfseries,
  bottomrule=0pt,
  boxrule=0pt,
  colframe=white,
  overlay unbroken and first={
  \draw[red!75!black,line width=3pt]
    ([xshift=5pt]frame.north west) -- 
    (frame.north west) -- 
    (frame.south west);
  \draw[red!75!black,line width=3pt]
    ([xshift=-5pt]frame.north east) -- 
    (frame.north east) -- 
    (frame.south east);
  },
  overlay unbroken app={
  \draw[red!75!black,line width=3pt,line cap=rect]
    (frame.south west) -- 
    ([xshift=5pt]frame.south west);
  \draw[red!75!black,line width=3pt,line cap=rect]
    (frame.south east) -- 
    ([xshift=-5pt]frame.south east);
  },
  overlay middle and last={
  \draw[red!75!black,line width=3pt]
    (frame.north west) -- 
    (frame.south west);
  \draw[red!75!black,line width=3pt]
    (frame.north east) -- 
    (frame.south east);
  },
  overlay last app={
  \draw[red!75!black,line width=3pt,line cap=rect]
    (frame.south west) --
    ([xshift=5pt]frame.south west);
  \draw[red!75!black,line width=3pt,line cap=rect]
    (frame.south east) --
    ([xshift=-5pt]frame.south east);
  },
}

\begin{document}


\texttt{R} inicia a principios de los noventas en la Universidad de
Auckland en Nueva Zelanda. Ross Ihaka, profesor del departamento de
estadística, pensaba que debía existir una alternativa superior para el
análisis de datos realizado por los alumnos, que utilizaban lo que él
llamaba \emph{programas viejos y cuchos}. Robert Gentleman le sugiere a
Ross escribir un software cuya ambición inicial era poder enseñar sus
cursos de licenciatura de primer año. Así, en 1991 generan una
estructura básica a través de la cuál sus estudiantes podían hacer
análisis de datos y producir modelos gráficos de la información. Lo
bautizan \texttt{R} por sus iniciales \parencite{rorigins}.

Ross y Robert no comercializan el software sino que lo ponen a
disposición de otros interesados. Ross ha expresado que \texttt{R}
cambió su opinión acerca de la humanidad pues es el resultado del
trabajo de muchos que no reciben ingresos o reconocimiento por el mismo
\parencite{rorigins}. En 1996, presentan \texttt{R} en un paper
introductorio \parencite{ihaka1996r}.

A partir de entonces, \texttt{R} ha crecido en forma importante. Entre
los contribuidores actuales más relevantes se encuentra Hadley Wickham,
alumno de licenciatura en el departamento de estadística de la
Universidad de Auckland cuando \texttt{R} se encontraba en desarrollo.
En la gráfica siguiente, se muestran las descargas anuales de paquetes
de \texttt{R} del 2012 al 2016 del espejo de
RStudio\footnote{Estos números representan únicamente una fracción de las descargas de \texttt{R} en el mundo pues existen múltiples espejos del software de donde es posible realizar la descarga. Los datos son tomados de \textcite{cranlogs}}.

\begin{figure}
\centering
\includegraphics{introduccion_files/figure-latex/unnamed-chunk-1-1.pdf}
\caption{Descargas anuales del espejo de RStudio de paquetes de R de
2012 a 2016 y descargas de R para 2015 y 2016 (en millones).}
\end{figure}

En el 2016 \texttt{R} fue descargado 670,705 veces. El aumento en la
popularidad de \texttt{R} no es el único elemento por el cuál \texttt{R}
es un lenguaje valioso. Sin embargo, el que sea un lenguaje comúnmente
enseñado en universidades y utilizado en empresas, lo convierte en una
habilidad con considerable valor de mercado.

En la encuesta de \texttt{Stackoverflow}, \texttt{R} se encuentra en el
lugar séptimo de los mejores pagados para los desarrolladores cuya
ocupación es matemáticas, superando a \texttt{Python} y a \texttt{SQL}
\parencite[][Top paying tech per occupation, mathematics]{stackoverflowsurvey16}.
En cuanto a las tecnologías más populares por tipo de desarrollador que
declara dedicarse a matemáticas y datos, \texttt{R} está en el sexto
lugar, el primer lugar lo tiene \texttt{python}, seguido de \texttt{SQL}
\parencite[][Most Popular Technologies per Dev Type, Math and Data]{stackoverflowsurvey16}.

Actualmente, \texttt{R}, \texttt{python} y \texttt{SQL} se encuentran
entre las herramientas más populares tanto entre desarrolladores como
empresas, aunque no son las únicas. La decisión de aprender alguno de
estos lenguajes depende de muchos factores, entre ellos cuán natural
resulta la interacción individual con cada cuál, el lenguaje preferido
en el grupo de trabajo particular y el tipo de análisis que se requiere
realizar en el día a día. Escapa del objetivo de este manual el realizar
una comparación exhaustiva de tecnologías pero se recomienda tener en
cuenta que cada herramienta tiene una especialidad específica y,
particularmente en un ambiente de producción, es necesario tener esto en
consideración.

\texttt{R} es un excelente lenguaje para aprender ciencia de datos; de
hecho en \textcite{cran} se describe a \texttt{R} como un proyecto para
estadística computacional. Esto lo convierte en un lenguaje único pues
fue construido por estadísticos y diseñado para realizar análisis de
datos.

Su uso generalizado en la comunidad estadística tiene la ventaja de que
casi cualquier prueba o técnica estadística puede ser encontrada en
algún paquete de \texttt{R} \parencite{recommendr}. Además, existe una
documentación extensa y estandarizada que facilita su uso.

Aunque el material para aprender \texttt{R} es amplio y hay una
comunidad mundial muy activa que constantemente produce nuevos recursos,
existen pocas referencias que faciliten iniciar su aprendizaje para
hispanoparlantes. En general, la documentación, listas de distribución,
libros y tutoriales están escritos en inglés.

Este manual tiene como objetivo guiar a principiantes en programación
que tienen una formación previa como analistas de datos. El enfoque
principal es el de facilitar de ejemplos que permitan al analista
traducir la manipulación de datos que ya saben realizar en otro ambiente
a \texttt{R}.

El manual se estructura como sigue: en el capítulo 2, se introducen
elementos básicos para poder iniciar el trabajo en \texttt{R}. Se
especifica cómo instalar el software, se recomienda utilizar un editor
especializado, así como paquetes útiles para diferentes tareas. En
particular, se explica cómo guardar código de manera que otras personas
puedan ejecutarlo y cómo realizar documentos reproducibles. Por último,
se explica cómo accesar a la ayuda y documentación, así como la forma en
la que puede optimizarse su funcionamiento. Este capítulo actúa más como
una referencia general para poder realizar el trabajo en el ambiente.

En el capítulo 3, se introducen las funciones, las estructuras de datos
y las estructuras de control disponibles en el lenguaje. El capítulo 4,
explica como operar los objetos y estructuras detallados en el capítulo
anterior, proporcionando múltiples ejemplos y ejercicios para
familiarizar al lector con el lenguaje.

El capítulo 5, detalla las herramientas básicas para poder realizar un
proyecto de datos en \texttt{R}. Las herramientas que se desarrollan en
este capítulo permiten iterar sobre parte del ciclo de un proyecto de
datos: importación de datos al ambiente, manipulación, limpieza y
visualización de los mismos. Éstas herramientas permiten operar sobre
los objetos introducidos en el capítulo 3 en una forma eficiente, fácil
de aprender, fácil de leer y que permite que el usuario realice
manipulaciones de datos complejas que le permitirán, a su vez, utilizar
todas las herramientas de modelado que \texttt{R} posee que necesitan
como insumo datos limpios y preparados en una forma específica.

Cada capítulo incluye ejercicios y respuestas a los mismos; al final se
recomienda material adicional para repasar los conceptos estudiados. El
material se encuentra disponible electrónicamente en
\url{https://github.com/animalito/aprendeR}. Para facilitar el
aprendizaje, se recomienda descargar los materiales o clonar el
repositorio, esto permite revisar el material y el código desde el
ambiente local evitando copiar y pegar el mismo para su ejecución.


\end{document}
 % antescedentes, situacion actual & objetivos

\input{Chapters/marco_conceptual}

\input{Chapters/pot}

\input{Chapters/solicitudes}
\graphicspath{{rgenerated/}}
\input{rgenerated/infomex}

\input{Chapters/rr}
\graphicspath{{rgenerated/}}
\input{rgenerated/rr}

\input{Chapters/usuarios}
\graphicspath{{rgenerated/}}
\input{rgenerated/perfil_usuario}

\input{Chapters/ejercicio_del_dai}

\input{Chapters/recomendaciones}

\input{Chapters/herramienta}

\input{Chapters/arquitectura}

%\input{Chapters/Chapter2} % Background Theory 

%\input{Chapters/Chapter3} % Experimental Setup

%\input{Chapters/Chapter4} % Experiment 1

%\input{Chapters/Chapter5} % Experiment 2

%\input{Chapters/Chapter6} % Results and Discussion

%\input{Chapters/Chapter7} % Conclusion

%% ----------------------------------------------------------------
% Now begin the Appendices, including them as separate files

\addtocontents{toc}{\vspace{2em}} % Add a gap in the Contents, for aesthetics

\appendix % Cue to tell LaTeX that the following 'chapters' are Appendices

% \input{Appendices/AppendixA}	% Appendix Title

%\input{Appendices/AppendixB} % Appendix Title

%\input{Appendices/AppendixC} % Appendix Title

\addtocontents{toc}{\vspace{2em}}  % Add a gap in the Contents, for aesthetics
\backmatter

%% ----------------------------------------------------------------
\label{Bibliography}
\lhead{\emph{Bibliography}}  % Change the left side page header to "Bibliography"
%\bibliographystyle{unsrtnat}  % Use the "unsrtnat" BibTeX style for formatting the Bibliography
%\bibliography{Bibliography}  % The references (bibliography) information are stored in the file named "Bibliography.bib"
\printbibliography

\end{document}  % The End
%% ----------------------------------------------------------------
